\documentclass[11pt,letterpaper]{article}
\usepackage[utf8]{inputenc}
\usepackage[T1]{fontenc}
\usepackage[activeacute,spanish]{babel}
\usepackage[left=18mm,right=18mm,top=21mm,bottom=21mm,letterpaper]{geometry}%
\usepackage{helvet}
\usepackage{tikz}
\usepackage{amsmath,amsfonts,amssymb,commath}
\usepackage{graphicx}
\usepackage{color}
\usepackage{xcolor}
\usepackage{verbatim}
\usepackage{tabls}
\usepackage[space]{grffile}
\usepackage{url}
\usepackage{listings}
%\usepackage{circuitikz}
\usepackage{siunitx}
\usepackage{fancyhdr}   
\pagestyle{fancy}
\usepackage{multicol,multirow}
\usepackage{textcomp}
\usepackage{booktabs}
\usepackage[colorlinks=true,urlcolor=blue,linkcolor=black,citecolor=black]{hyperref} 
\usepackage{pdfpages}   %incluir paginas de pdf externo, para los anexos
\usepackage{appendix}
\usepackage{caption}
\usepackage{subcaption}  
\usepackage{apacite}
\usepackage{natbib}
\usepackage{rotating}
\usepackage[section]{placeins}
\usepackage{titlesec}
\usepackage{wrapfig}

\lhead{ Laboratorio Sistemas Incrustados}
\chead{}
\rhead{Laboratorio 1}   % Aquí va el numero de experimento, al igual que en el titulo
\lfoot{Escuela de Ingeniería Eléctrica}
\cfoot{\thepage}
\rfoot{Universidad de Costa Rica}

\titleformat*{\section}{\large\bfseries}
\titleformat*{\subsection}{\bfseries}

\pagenumbering{Roman}
%------------------------------------------------------------

\author{ Daniel García Vaglio, B42781\\ Esteban Zamora Alvarado, B47769 \\
\\ {\small Grupo 01}\\ Profesor: Esteban Ortiz \vspace*{3.0in}}

\begin{document}
%Header-Make sure you update this information!!!!
\noindent
\large\textbf{Laboratorio II: reporte} \hfill Laboratorio de sistemas Incrustados \\
\normalsize Esteban Zamora Alvarado \hfill Carné: B47769 \\
Daniel Esteban García Vaglio \hfill Carné: B42781 \\
Profesor: Esteban Ortiz  \hfill Fecha: 19-Octubre-2017 \\

\section{Problema a Resolver}

Utilizar la tarjeta MSP432 junto con el kit de expansión para desarrollar un indicador
de atitud. Este debe ser presentado al usuario por medio de la pantalla LCD que ya está incorporada
al kit de expansión. Debe ser capaz de indicar el pitch y el roll con una velocidad máxima de 90
grados por segundo y un rango máximo de 180 grados.  

\section{Contexto de Mercado}


\section{Solución Propuesta}

Como se deben realizar varias tareas de forma simultánea se propone la utilización de un scheduler
simple. Esto es un programa que se encarga de decidir qué tareas se deben ser ejecutadas y de la
sincronización general de las mismas. De esta manera se garantiza el cumplimiento de las
restricciones de temporización de cada proceso. Este scheduler permite el paso de mensajes entre las
tareas que debe sincronizar y además proporciona un método para reservar memoria para la
transferencia de datos por mensajes. También soporta 2 tipos de tareas, las que son periódicas, que
el scheduler debe ejecutar cada cierto tiempo, y las que son one shot, que el scheduler debe
ejecutar cada vez que sucede cierto evento.

El sistema que soluciona este laboratorio toma los datos de la dirección del vector de gravedad y a
partir de ellos calcula el Pitch y el Roll. Luego eso lo transforma en una ecuación lineal que es la
que define cómo se deben dibujar los colores en la pantalla LCD. Sobre la ecuación se dibuja azul, y
debajo se dibuja café. 

\section{Configuración de periféricos}

\subsection{Timer}

\subsection{Sensor de luz con I2C}


\subsection{ ADC}

\subsection{Acelerómetro}

%--------------------------------
\section{Implementación}
%-------------------------------

\subsection{Scheduler}


\subsection{Task: de cálculo de ángulos}

\subsection{Task: LCD issue}

\subsection{Task: LCD draw}

\subsection{Driver optimizado}

%-------------------------------
\section{Problemas encontrados y posibles mejoras}
% -------------------------------


%----------------------
\textbf{ Bibliografía}
% ----------------------

Technical Reference manual, MSP432P4XX Simple Link Microcontrollers. Texas Instruments. Marzo 2017.

MSP432 Peripheral Driver Library. Users Guide. Texas Instruments. 2015.

BOOSTXL-EDUMKII Educational BoosterPack, Module Mark II. Texas Instruments. Marzo 2017.


%\bibliographystyle{apacite}
%\bibliography{../general/biblografia.bib}
%---------------------------------------------------------------------------------------------------------------
\end{document}