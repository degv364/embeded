\documentclass[11pt,letterpaper]{article}
\usepackage[utf8]{inputenc}
\usepackage[T1]{fontenc}
\usepackage[activeacute,spanish]{babel}
\usepackage[left=18mm,right=18mm,top=21mm,bottom=21mm,letterpaper]{geometry}%
\usepackage{helvet}
\usepackage{tikz}
\usepackage{amsmath,amsfonts,amssymb,commath}
\usepackage{graphicx}
\usepackage{color}
\usepackage{xcolor}
\usepackage{verbatim}
\usepackage{tabls}
\usepackage[space]{grffile}
\usepackage{url}
\usepackage{listings}
%\usepackage{circuitikz}
\usepackage{siunitx}
\usepackage{fancyhdr}   
\pagestyle{fancy}
\usepackage{multicol,multirow}
\usepackage{textcomp}
\usepackage{booktabs}
\usepackage[colorlinks=true,urlcolor=blue,linkcolor=black,citecolor=black]{hyperref} 
\usepackage{pdfpages}   %incluir paginas de pdf externo, para los anexos
\usepackage{appendix}
\usepackage{caption}
\usepackage{subcaption}  
\usepackage{apacite}
\usepackage{natbib}
\usepackage{rotating}
\usepackage[section]{placeins}
\usepackage{titlesec}
\usepackage{wrapfig}

\lhead{ Laboratorio Sistemas Incrustados}
\chead{}
\rhead{Laboratorio 1}   % Aquí va el numero de experimento, al igual que en el titulo
\lfoot{Escuela de Ingeniería Eléctrica}
\cfoot{\thepage}
\rfoot{Universidad de Costa Rica}

\titleformat*{\section}{\large\bfseries}
\titleformat*{\subsection}{\bfseries}

\pagenumbering{Roman}
%------------------------------------------------------------

\author{ Daniel García Vaglio, B42781\\ Esteban Zamora Alvarado, B47769 \\
\\ {\small Grupo 01}\\ Profesor: Esteban Ortiz \vspace*{3.0in}}

\begin{document}
%Header-Make sure you update this information!!!!
\noindent
\large\textbf{Laboratorio II: reporte} \hfill Laboratorio de sistemas Incrustados \\
\normalsize Esteban Zamora Alvarado \hfill Carné: B47769 \\
Daniel Esteban García Vaglio \hfill Carné: B42781 \\
Profesor: Esteban Ortiz  \hfill Fecha: 19-Octubre-2017 \\

\section{Problema a Resolver}

Utilizar la tarjeta MSP432 junto con el kit de expansión para desarrollar un
indicador de actitud. Este debe ser presentado al usuario por medio de la
pantalla LCD que ya está incorporada al kit de expansión. Debe ser capaz de
indicar el pitch y el roll con una velocidad mínima de 90 grados por segundo y
un rango máximo de 180 grados.

\section{Contexto de Mercado}
Un indicator de actitud, es utilizado en aeronaves para informarle al piloto
sobre la orientación de la aeronave con respecto a la Tierra. De esta forma,
indica visualmente los ángulos de Pitch y Roll (Bank) y es fundamental para el
vuelo bajo condiciones meteorológicas, que pueden llegar a distorsionar el
sentido de orientación del piloto. Generalmente, estos están basados en un
giroscopio, el cual tiene dos grados de libertad para poder representar los dos
ángulos ya indicados, lo cual se visualiza mediante una línea de horizonte y
unas escalas de medición en la pantalla. Debido a los requerimientos de
confiabilidad para la industria aeronáutica, estos equipos pueden llegar a tener
un alto precio. Ejemplos de este tipo de indicadores corresponden a la serie
RCA26 EK de Kelly Manufacturing Co., en donde estos se encuentran en un rango
de precios de alrededor de \$2000.

El sistema diseñado, debido a sus características constructivas como el hardware
disponible (limitada precisión de los sensores y de velocidad de escritura a la
LCD), solo simula de forma aproximada el comportamiento de un indicador de
actitud, por lo que no se podría llegar a pensar en la utilización del mismo en
una aplicación aeronáutica. Sin embargo, debido a que el mismo se basa en
scheduling corresponde a un sistema versátil y extensible que se podría emplear
en aplicaciones relacionadas con entretenimiento, tales como videojuegos,
simulación y realidad virtual, agregando otros tasks que implementen la
funcionalidad requerida.

\section{Solución Propuesta}

Como se deben realizar varias tareas de forma simultánea se requiere la
utilización de un scheduler simple. Esto es un programa que se encarga de
decidir qué tareas se deben ser ejecutadas y de la sincronización general de las
mismas. De esta manera se garantiza el cumplimiento de las restricciones de
temporización de cada proceso. Este scheduler permite el paso de mensajes entre
las tareas que debe sincronizar y además proporciona un método para reservar
memoria para la transferencia de datos por mensajes. El diseño planteado soporta
3 tipos de tareas, las que son periódicas, que el scheduler debe ejecutar cada
cierto tiempo, las que son one shot, que el scheduler debe ejecutar cada vez
que sucede cierto evento y las de interrupción, que se emplean para poder enviar
mensajes a otras tareas o al scheduler dentro de rutinas de atención a interrupción.

El sistema que soluciona este laboratorio toma los datos de la dirección del
vector de gravedad y a partir de ellos calcula los ángulos de Pitch y el Roll.
Luego eso lo transforma en una ecuación lineal que es la que define cómo se
deben dibujar los colores en la pantalla LCD. Sobre la ecuación se dibuja azul
(cielo), y debajo se dibuja café (tierra).

\section{Configuración de periféricos}
Para la implementación del sistema propuesto se utilizaron los periféricos de la
MSP432 correspondientes al Timer32, el ADC14 y los GPIOs para los LEDs. Los
mismos fueron configurados mediante la API de Driverlib.
\subsection{Timer32}
El módulo del Timer32 se utiliza en el diseño para establecer la frecuencia del
tick del scheduler, el cual define cada cuanto tiempo se comienza a ejecutar un
nuevo frame, actualizando un contador de ticks en su rutina de interrupción
correspondiente. Para esto el mismo se configura en modo periódico, tal que al
terminar el conteo descendente hasta 0 se reinicie el valor máximo de conteo, el
cual a su vez permite establecer la cantidad de interrupciones por segundo en
conjunto con la frecuencia del MCLK (con preescalado de 16). En función de las
tareas a realizar en programa final, se logró llegar a una frecuencia de tick de
310 Hz.

También, se emplea el otro Timer32 para establecer la frecuencia de muestreo del
ADC, tal como se explica a continuación.

\subsection{ADC/Acelerómetro}
Para capturar los datos del acelerómetro, el cual funciona de manera analógica,
es necesario utilizar el módulo ADC14 del MSP432. De esta forma, para capturar
los valores de los 3 canales del acelerómetro se configura el ADC en el modo de
secuencia múltiple, de manera que se lleve a cabo la conversión dichos canales
de forma secuencial pero rápida (el muestreo de un canal se configura tal que
ocurre apenas el ADC termine de realizar la conversión del anterior), guardando
los resultados en 3 registros del ADC (MEM0, MEM1 y MEM2). Este muestreo de la
secuencia de canales se activa mediante una señal de software, de manera que una
interrupción de un Timer32 establece la frecuencia promedio con la que se
muestrea el acelerómetro, ya que el tiempo de muestreo de los 3 canales es muy
pequeño con respecto al tiempo entre interrupciones del Timer32. En la
configuración final implementada se utilizó una frecuencia de 310 Hz para el
Timer32.

Además de esto se requirió configurar la interrupción del ADC de manera que se
active cuando se termina la conversión del último canal (MEM2), lo cual indica
que los valores de los registros del ADC se encuentran actualizados y deben ser
capturados por el programa principal.

También, cabe destacar que se requirió configurar los pines GPIO P4.0, P4.2 y P6.1
como entradas para así poder capturar los valores del acelerómetro, así como
mapear dichos pines como entradas de los canales del ADC.

\subsection{LEDs}
Para poder reconocer visualmente cuando se generaba una situación de error en el
software implementado (scheduler y tasks), así como cuando el tiempo de
ejecución de las tareas superaba el tiempo máximo del frame, se utilizaron dos
de los LEDs RGB de la tarjeta de desarrollo. Para esto los mismos se configuran
como salidas GPIO y se activan por software en una situación de error de
software (rojo) o de timeout (amarillo).

%--------------------------------
\section{Implementación}
%-------------------------------
La implementación del código se encuentra disponible en
\url{https://github.com/degv364/embeded}.

\subsection{Scheduler}
\label{sec:scheduler}
El scheduler tiene como miembro un arreglo de tasks (sección \ref{sec:tasks}).
Estas son todas las tareas que puede ejecutar pero no necesariamente las ejecuta
todas en cada frame. La prioridad de ejecución se define por el nombre del task.
Esto se hace aprovechando el hecho que el nombre se implementa como un enum y
por lo tanto ya tiene un orden intrínseco, cuanto menor sea el valor del enum,
mayor es la prioridad. El nombre del task define la posición dentro del arreglo
de tasks que tiene el scheduler. De esta manera se optimiza bastante los
procesos de búsqueda y ordenamiento de tasks ($O(1)$). En particular, el
scheduler tiene asignado la primera posición del enum para poder indicarlo como
destinatario de mensajes.

El scheduler le agrega a cada task información extra. Primero agrega un contador
que en el caso de las tareas periódicas indica cuántos frames faltan para que la
tarea se deba ejecutar de nuevo. El otro dato es una bandera que indica si el
task debe ser ejecutado al siguiente frame.

En cada frame se itera por las tareas y se ejecutan en caso que su bandera de
ejecución esté activada. Cabe destacar que como los tasks se indexan por nombre,
que a su vez indica la prioridad. Siempre se ejecutan primero los tasks de mayor
prioridad. También durante el postamble se atienden los tasks de mayor prioridad
primero.

Las tareas misceláneas del scheduler se ejecutan durante el postamble. Primero
el scheduler itera por todos los tasks y los mensajes que estos colocaron en su
message\_queue de salida, se pasan a los tasks destinatarios. Cabe destacar que
cada task escribe el contenido de dichos mensajes en un espacio de memoria
brindado mediante un puntero por parte del scheduler que gestiona un espacio de
memoria estático tipo ``heap'' para distribuir entre las tareas.

En caso de que los mensajes sean para el scheduler, se guardan en el
message\_queue interno al scheduler. También se actualizan los contadores para
ejecución de los tasks periódicos, y se les activa la bandera de ejecución en
caso que sea necesario. Luego se procesan los mensajes internos, tales como los
que le indican al scheduler que agregue un task específico (tipo one-shot) a la
ejecución del siguiente frame.


\subsection{Tasks}
\label{sec:tasks}

Todos los tasks tienen un nombre, que define su prioridad. Además, en el caso de
los tasks periódicos se les define un periodo. Los message\_queque de salida y
de entrada se utilizan para enviar y recibir mensajes respectivamente.

El método de setup se encarga de inicializar varios miembros de la clase. Además
se solicita al heap una porción de memoria para ser utilizada para el paso de
mensajes. Durante el método de run, se revisan los mensajes entrantes y se
atienden, luego se ejecutan las tareas propias del task.

\subsubsection{Task: ADC\_IRQ}
\label{sec:adc_irq}

Este task se encarga de gestionar la transmisión de los datos del acelerómetro a
la tarea \texttt{CALC\_HORIZON}, de manera que no se ejecuta su método
\texttt{Run}, sino se creó para poder asignarle un espacio de memoria en el heap
que le permitiera enviar mensajes. Además, se encarga de indicarle al scheduler
que ejecute la tarea \texttt{CALC\_HORIZON} (tipo one-shot).

\subsubsection{Task: CALC\_HORIZON}
\label{sec:calc_horizon}
Esta es una tarea tipo one-shot que se ejecuta cuando el scheduler recibe el
mensaje por parte de la rutina de atención a interrupciones del ADC14. Esta se
encarga de tomar los datos del acelerómetro y generar la información relacionada
a los ángulos de Pitch (nivel de horizonte) y Roll (pendiente del horizonte).
Del vector de gravedad se obtiene el ángulo de pitch como $P = atan(y/z)$, que
luego es mapeado linealmente al nivel de horizonte en el eje Y de la LCD. La
pendiente debida al Roll se obtiene como $m = x/\sqrt{(y^2+z^2)}$. Los
resultados se envían a la tarea \texttt{LCD\_ISSUE}.

\subsubsection{Task: LCD\_ISSUE}
\label{sec:lcd_issue}

Como parte de las optimizaciones realizadas, se dividió la pantalla en 16
cuadrados iguales. Esto permite tener una tamaño fijo de frame aunque cambie el
porcentaje de la pantalla que se debe re-dibujar, aparte que permite reducir el
tamaño del frame. Este task periódico calcula a partir de la información de
Pitch y Roll, cuales son los cuadrados que deben ser re-dibujados, y genera una
lista que le pasa por medio de mensajes a \texttt{LCD\_DRAW} junto con la
información de Pitch y Roll. Solicita al scheduler ejecutar \texttt{LCD\_DRAW}
(tipo one-shot).

\subsubsection{Task: LCD\_DRAW}
\label{sec:lcd_draw}
Este task toma la lista de las coordenadas de los cuadros que le pasó
\texttt{LCD\_ISSUE} y va dibujando un cuadro de tamaño fijo por frame. Si aún no
se han dibujado todos los cuadros de la lista, entonces se le envía una
solicitud al scheduler para que en el siguiente frame se ejecute otra vez el
mismo task (auto-trigger), y de esta manera dibujar todos los cuadros
requeridos.

\subsection{Driver LCD optimizado}
Driverlib ofrece un API para controlar a la pantalla LCD, sin embargo según lo
observado no era lo suficientemente rápido para la aplicación requerida. Por
otra parte no ofrece funciones para dibujar líneas inclinadas, por lo que se
optó por implementar una versión optimizada del driver. Esta versión permite
dibujar cuadrados de $32\times 32$ píxeles, en donde hay dos sectores de color
(celeste y café), en función de la información de la recta definida por los
ángulos de Pitch y Roll. Al utilizar este driver optimizado y compilar con
optimizaciones del compilador, se logró aumentar la frecuencia de $135$ frames
por segundo a $330$.

%-------------------------------
\section{Problemas encontrados y posibles mejoras}
% -------------------------------

Actualmente se re-dibuja toda la pantalla cada vez que se ejecuta lcd\_issue.
Esto por problemas a la hora de identificar los cuadros que necesitan
actualización. Para arreglarlo se deben identificar adecuadamente las
condiciones de re-dibujado y una manera eficiente de computarlas.

Para ciertos ángulos de roll se obtienen resultados erráticos de Pitch. Esto
porque es posible que para el cálculo del mismo se generen indeterminaciones del
tipo $0/0$. Una manera de corregirlo es agregarle un pequeño factor de
aproximación que depende del eje $x$ de la aceleración, de manera que no solo
dependa de dos ejes, y entonces no se indefinan sus valores de cálculo.

En este momento solo se contempla una recta para el dibujado, pero si se quiere
hacer pitch y roll completo ($360$ grados) es necesario agregar rectas en $180$
y $-180$. Aunque matemáticamente son equivalentes, a la hora de implementarlo
proporcionan una manera cómoda de lidiar con las condiciones inversas (café
arriba, azul abajo) de dibujado de colores cuando la tarjeta se encuentra ``de
cabeza''.


%----------------------
\textbf{ Bibliografía}
% ----------------------

Technical Reference manual, MSP432P4XX Simple Link Microcontrollers. Texas Instruments. Marzo 2017.

MSP432 Peripheral Driver Library. Users Guide. Texas Instruments. 2015.

BOOSTXL-EDUMKII Educational BoosterPack, Module Mark II. Texas Instruments. Marzo 2017.


%\bibliographystyle{apacite}
%\bibliography{../general/biblografia.bib}
%---------------------------------------------------------------------------------------------------------------
\end{document}