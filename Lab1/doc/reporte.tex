\documentclass[12pt,letterpaper]{article}
\usepackage[utf8]{inputenc}
\usepackage[T1]{fontenc}
\usepackage[activeacute,spanish]{babel}
\usepackage[left=18mm,right=18mm,top=21mm,bottom=21mm,letterpaper]{geometry}%
\usepackage{helvet}
\usepackage{tikz}
\usepackage{amsmath,amsfonts,amssymb,commath}
\usepackage{graphicx}
\usepackage{color}
\usepackage{xcolor}
\usepackage{verbatim}
\usepackage{tabls}
\usepackage[space]{grffile}
\usepackage{url}
\usepackage{listings}
%\usepackage{circuitikz}
\usepackage{siunitx}
\usepackage{fancyhdr}   
\pagestyle{fancy}
\usepackage{multicol,multirow}
\usepackage{textcomp}
\usepackage{booktabs}
\usepackage[colorlinks=true,urlcolor=blue,linkcolor=black,citecolor=black]{hyperref} 
\usepackage{pdfpages}   %incluir paginas de pdf externo, para los anexos
\usepackage{appendix}
\usepackage{caption}
\usepackage{subcaption}  
\usepackage{apacite}
\usepackage{natbib}
\usepackage{rotating}
\usepackage[section]{placeins}

\lhead{ Laboratorio Sistemas Incrustados}
\chead{}
\rhead{Laboratorio 1}   % Aquí va el numero de experimento, al igual que en el titulo
\lfoot{Escuela de Ingeniería Eléctrica}
\cfoot{\thepage}
\rfoot{Universidad de Costa Rica}

\definecolor{vgreen}{RGB}{104,180,104}
\definecolor{vblue}{RGB}{49,49,255}
\definecolor{vorange}{RGB}{255,143,102}

\lstset{
  language=Verilog,
  basicstyle=\small,
  keywordstyle=\color{vblue},
  identifierstyle=\color{black},
  commentstyle=\color{vgreen},
  numbers=left,
  numberstyle={\tiny \color{black}},
  numbersep=10pt,
  tabsize=4
}

%-------------------------------------------------------------

%\renewcommand{\labelenumi}{\alph{enumi}.}
%\addto\captionsspanish{\renewcommand{\tablename}{Tabla}}                    % Cambiar nombre a tablas
%\addto\captionsspanish{\renewcommand{\listtablename}{Índice de tablas}}     % Cambiar nombre a lista de tablas
\pagenumbering{Roman}
%------------------------------------------------------------

\author{ Daniel García Vaglio, B42781\\ Esteban Zamora Alvarado, B47769 \\
\\ {\small Grupo 01}\\ Profesor: Esteban Ortiz \vspace*{3.0in}}


\title{Universidad de Costa Rica\\{\small Facultad de Ingeniería\\Escuela de Ingeniería Eléctrica\\
    Laboratorio Sistemas incrustados\\II ciclo 2017\\\vspace*{0.55in} Reporte de Laboratorio}\\ Práctica I \vspace*{1.1in}}
\date{20 de Setiembre de 2017}

\begin{document}


\pdfbookmark[1]{Portada}{portada}

\maketitle
\thispagestyle{empty}
\newpage


\setcounter{page}{1}
\tableofcontents
%\newpage
%\listoffigures
%\listoftables

\newpage

\pagenumbering{arabic}



\newpage
%--------------------------------
\section{Solución propuesta}
%-------------------------------

Para mejorar el desarrollo de la solución de este laboratorio, se tomó la decisión de dividir todo
el código en la parte que depende del hardware, y la que no depende del hardware. De esta manera se
pueden tener avances y pruebas sin necesidad de tener la tarjeta de desarrollo (dos integrantes con
una sola tarjeta). El código que depende del hardware se probó en la tarjeta directamente, por otra
parte el código que no es dependiente del hardware se probó utilizando el ``unit testing'' que
ofrece \textit{google-test}.

Otro criterio importante de diseño es que se modularizó el código, de manera que cada sensor se
maneja con su propia clase, y las partes que son independientes del hardware se separaron
también. Esto tiene la ventaja que permite tener código más entendible, y hace más fácil hacer
modificaciones del funcionamiento del proyecto. De hecho durante el desarrollo del proyecto se
hicieron varios cambios en el funcionamiento de ciertos módulos, y esto no afectó al resto de
módulos. Por ejemplo se cambió la manera en que se calcula si el segundo actual está sobre el
promedio establecido para encender la luz. Solo se hicieron cambios en uno de los módulos, y el
resto del proyecto no sufrió modificaciones.

Se tiene un loop de control principal, que al inicio se encarga de inicializar el hardware y los
periféricos y luego se encarga de iterar periódicamente y definir su comportamiento dependiendo de
las lecturas de los sensores. Es importante destacar que las lecturas de los sensores no se hace
dentro de este loop principal, sino que se manejan con sus propias interrupciones, por lo que fue
necesario crear el \textbf{struct de sensores} que se explica más adelante. 


\subsection{Código dependiente del hardware (HD)}



\subsection{Código independiente del hardware (HI)}

\begin{figure}
\centering
\scalebox{.5}{% Graphic for TeX using PGF
% Title: /home/daniel/repos/ucr/embeded/Lab1/doc/data_flow.dia
% Creator: Dia v0.97.3
% CreationDate: Wed Sep 20 13:35:17 2017
% For: daniel
% \usepackage{tikz}
% The following commands are not supported in PSTricks at present
% We define them conditionally, so when they are implemented,
% this pgf file will use them.
\ifx\du\undefined
  \newlength{\du}
\fi
\setlength{\du}{15\unitlength}
\begin{tikzpicture}
\pgftransformxscale{1.000000}
\pgftransformyscale{-1.000000}
\definecolor{dialinecolor}{rgb}{0.000000, 0.000000, 0.000000}
\pgfsetstrokecolor{dialinecolor}
\definecolor{dialinecolor}{rgb}{1.000000, 1.000000, 1.000000}
\pgfsetfillcolor{dialinecolor}
\definecolor{dialinecolor}{rgb}{1.000000, 1.000000, 1.000000}
\pgfsetfillcolor{dialinecolor}
\fill (8.695000\du,5.700000\du)--(8.695000\du,7.600000\du)--(13.200000\du,7.600000\du)--(13.200000\du,5.700000\du)--cycle;
\pgfsetlinewidth{0.100000\du}
\pgfsetdash{}{0pt}
\pgfsetdash{}{0pt}
\pgfsetmiterjoin
\definecolor{dialinecolor}{rgb}{0.000000, 0.000000, 0.000000}
\pgfsetstrokecolor{dialinecolor}
\draw (8.695000\du,5.700000\du)--(8.695000\du,7.600000\du)--(13.200000\du,7.600000\du)--(13.200000\du,5.700000\du)--cycle;
% setfont left to latex
\definecolor{dialinecolor}{rgb}{0.000000, 0.000000, 0.000000}
\pgfsetstrokecolor{dialinecolor}
\node at (10.947500\du,6.845000\du){Time};
\definecolor{dialinecolor}{rgb}{1.000000, 1.000000, 1.000000}
\pgfsetfillcolor{dialinecolor}
\fill (8.397500\du,10.300000\du)--(8.397500\du,12.200000\du)--(13.492500\du,12.200000\du)--(13.492500\du,10.300000\du)--cycle;
\pgfsetlinewidth{0.100000\du}
\pgfsetdash{}{0pt}
\pgfsetdash{}{0pt}
\pgfsetmiterjoin
\definecolor{dialinecolor}{rgb}{0.000000, 0.000000, 0.000000}
\pgfsetstrokecolor{dialinecolor}
\draw (8.397500\du,10.300000\du)--(8.397500\du,12.200000\du)--(13.492500\du,12.200000\du)--(13.492500\du,10.300000\du)--cycle;
% setfont left to latex
\definecolor{dialinecolor}{rgb}{0.000000, 0.000000, 0.000000}
\pgfsetstrokecolor{dialinecolor}
\node at (10.945000\du,11.445000\du){Light Sensor};
\definecolor{dialinecolor}{rgb}{1.000000, 1.000000, 1.000000}
\pgfsetfillcolor{dialinecolor}
\fill (8.631250\du,12.550000\du)--(8.631250\du,14.450000\du)--(13.458750\du,14.450000\du)--(13.458750\du,12.550000\du)--cycle;
\pgfsetlinewidth{0.100000\du}
\pgfsetdash{}{0pt}
\pgfsetdash{}{0pt}
\pgfsetmiterjoin
\definecolor{dialinecolor}{rgb}{0.000000, 0.000000, 0.000000}
\pgfsetstrokecolor{dialinecolor}
\draw (8.631250\du,12.550000\du)--(8.631250\du,14.450000\du)--(13.458750\du,14.450000\du)--(13.458750\du,12.550000\du)--cycle;
% setfont left to latex
\definecolor{dialinecolor}{rgb}{0.000000, 0.000000, 0.000000}
\pgfsetstrokecolor{dialinecolor}
\node at (11.045000\du,13.695000\du){Microphone};
\definecolor{dialinecolor}{rgb}{1.000000, 1.000000, 1.000000}
\pgfsetfillcolor{dialinecolor}
\fill (8.500000\du,8.050000\du)--(8.500000\du,9.950000\du)--(13.190000\du,9.950000\du)--(13.190000\du,8.050000\du)--cycle;
\pgfsetlinewidth{0.100000\du}
\pgfsetdash{}{0pt}
\pgfsetdash{}{0pt}
\pgfsetmiterjoin
\definecolor{dialinecolor}{rgb}{0.000000, 0.000000, 0.000000}
\pgfsetstrokecolor{dialinecolor}
\draw (8.500000\du,8.050000\du)--(8.500000\du,9.950000\du)--(13.190000\du,9.950000\du)--(13.190000\du,8.050000\du)--cycle;
% setfont left to latex
\definecolor{dialinecolor}{rgb}{0.000000, 0.000000, 0.000000}
\pgfsetstrokecolor{dialinecolor}
\node at (10.845000\du,9.195000\du){Button};
\definecolor{dialinecolor}{rgb}{1.000000, 1.000000, 1.000000}
\pgfsetfillcolor{dialinecolor}
\fill (8.400000\du,14.850000\du)--(8.400000\du,18.150000\du)--(13.550000\du,18.150000\du)--(13.550000\du,14.850000\du)--cycle;
\pgfsetlinewidth{0.100000\du}
\pgfsetdash{}{0pt}
\pgfsetdash{}{0pt}
\pgfsetmiterjoin
\definecolor{dialinecolor}{rgb}{0.000000, 0.000000, 0.000000}
\pgfsetstrokecolor{dialinecolor}
\draw (8.400000\du,14.850000\du)--(8.400000\du,18.150000\du)--(13.550000\du,18.150000\du)--(13.550000\du,14.850000\du)--cycle;
% setfont left to latex
\definecolor{dialinecolor}{rgb}{0.000000, 0.000000, 0.000000}
\pgfsetstrokecolor{dialinecolor}
\node at (10.975000\du,16.295000\du){Microphone};
% setfont left to latex
\definecolor{dialinecolor}{rgb}{0.000000, 0.000000, 0.000000}
\pgfsetstrokecolor{dialinecolor}
\node at (10.975000\du,17.095000\du){FIFO};
\definecolor{dialinecolor}{rgb}{1.000000, 1.000000, 1.000000}
\pgfsetfillcolor{dialinecolor}
\fill (23.050000\du,9.550000\du)--(23.050000\du,12.800000\du)--(27.400000\du,12.800000\du)--(27.400000\du,9.550000\du)--cycle;
\pgfsetlinewidth{0.100000\du}
\pgfsetdash{}{0pt}
\pgfsetdash{}{0pt}
\pgfsetmiterjoin
\definecolor{dialinecolor}{rgb}{0.000000, 0.000000, 0.000000}
\pgfsetstrokecolor{dialinecolor}
\draw (23.050000\du,9.550000\du)--(23.050000\du,12.800000\du)--(27.400000\du,12.800000\du)--(27.400000\du,9.550000\du)--cycle;
% setfont left to latex
\definecolor{dialinecolor}{rgb}{0.000000, 0.000000, 0.000000}
\pgfsetstrokecolor{dialinecolor}
\node at (25.225000\du,10.970000\du){State};
% setfont left to latex
\definecolor{dialinecolor}{rgb}{0.000000, 0.000000, 0.000000}
\pgfsetstrokecolor{dialinecolor}
\node at (25.225000\du,11.770000\du){Machine};
\definecolor{dialinecolor}{rgb}{1.000000, 1.000000, 1.000000}
\pgfsetfillcolor{dialinecolor}
\fill (28.600000\du,10.175000\du)--(28.600000\du,12.075000\du)--(31.812500\du,12.075000\du)--(31.812500\du,10.175000\du)--cycle;
\pgfsetlinewidth{0.100000\du}
\pgfsetdash{}{0pt}
\pgfsetdash{}{0pt}
\pgfsetmiterjoin
\definecolor{dialinecolor}{rgb}{0.000000, 0.000000, 0.000000}
\pgfsetstrokecolor{dialinecolor}
\draw (28.600000\du,10.175000\du)--(28.600000\du,12.075000\du)--(31.812500\du,12.075000\du)--(31.812500\du,10.175000\du)--cycle;
% setfont left to latex
\definecolor{dialinecolor}{rgb}{0.000000, 0.000000, 0.000000}
\pgfsetstrokecolor{dialinecolor}
\node at (30.206250\du,11.320000\du){Lamps};
\definecolor{dialinecolor}{rgb}{1.000000, 1.000000, 1.000000}
\pgfsetfillcolor{dialinecolor}
\fill (16.100000\du,5.500000\du)--(16.100000\du,16.850000\du)--(21.800000\du,16.850000\du)--(21.800000\du,5.500000\du)--cycle;
\pgfsetlinewidth{0.100000\du}
\pgfsetdash{}{0pt}
\pgfsetdash{}{0pt}
\pgfsetmiterjoin
\definecolor{dialinecolor}{rgb}{0.000000, 0.000000, 0.000000}
\pgfsetstrokecolor{dialinecolor}
\draw (16.100000\du,5.500000\du)--(16.100000\du,16.850000\du)--(21.800000\du,16.850000\du)--(21.800000\du,5.500000\du)--cycle;
% setfont left to latex
\definecolor{dialinecolor}{rgb}{0.000000, 0.000000, 0.000000}
\pgfsetstrokecolor{dialinecolor}
\node at (18.950000\du,11.370000\du){};
\definecolor{dialinecolor}{rgb}{1.000000, 1.000000, 1.000000}
\pgfsetfillcolor{dialinecolor}
\fill (16.700000\du,7.950000\du)--(16.700000\du,9.850000\du)--(21.200000\du,9.850000\du)--(21.200000\du,7.950000\du)--cycle;
\pgfsetlinewidth{0.100000\du}
\pgfsetdash{}{0pt}
\pgfsetdash{}{0pt}
\pgfsetmiterjoin
\definecolor{dialinecolor}{rgb}{0.000000, 0.000000, 0.000000}
\pgfsetstrokecolor{dialinecolor}
\draw (16.700000\du,7.950000\du)--(16.700000\du,9.850000\du)--(21.200000\du,9.850000\du)--(21.200000\du,7.950000\du)--cycle;
% setfont left to latex
\definecolor{dialinecolor}{rgb}{0.000000, 0.000000, 0.000000}
\pgfsetstrokecolor{dialinecolor}
\node at (18.950000\du,9.095000\du){Time};
\definecolor{dialinecolor}{rgb}{1.000000, 1.000000, 1.000000}
\pgfsetfillcolor{dialinecolor}
\fill (16.605000\du,10.300000\du)--(16.605000\du,12.200000\du)--(21.295000\du,12.200000\du)--(21.295000\du,10.300000\du)--cycle;
\pgfsetlinewidth{0.100000\du}
\pgfsetdash{}{0pt}
\pgfsetdash{}{0pt}
\pgfsetmiterjoin
\definecolor{dialinecolor}{rgb}{0.000000, 0.000000, 0.000000}
\pgfsetstrokecolor{dialinecolor}
\draw (16.605000\du,10.300000\du)--(16.605000\du,12.200000\du)--(21.295000\du,12.200000\du)--(21.295000\du,10.300000\du)--cycle;
% setfont left to latex
\definecolor{dialinecolor}{rgb}{0.000000, 0.000000, 0.000000}
\pgfsetstrokecolor{dialinecolor}
\node at (18.950000\du,11.445000\du){Button};
\definecolor{dialinecolor}{rgb}{1.000000, 1.000000, 1.000000}
\pgfsetfillcolor{dialinecolor}
\fill (16.402500\du,12.400000\du)--(16.402500\du,14.300000\du)--(21.497500\du,14.300000\du)--(21.497500\du,12.400000\du)--cycle;
\pgfsetlinewidth{0.100000\du}
\pgfsetdash{}{0pt}
\pgfsetdash{}{0pt}
\pgfsetmiterjoin
\definecolor{dialinecolor}{rgb}{0.000000, 0.000000, 0.000000}
\pgfsetstrokecolor{dialinecolor}
\draw (16.402500\du,12.400000\du)--(16.402500\du,14.300000\du)--(21.497500\du,14.300000\du)--(21.497500\du,12.400000\du)--cycle;
% setfont left to latex
\definecolor{dialinecolor}{rgb}{0.000000, 0.000000, 0.000000}
\pgfsetstrokecolor{dialinecolor}
\node at (18.950000\du,13.545000\du){Light Sensor};
\definecolor{dialinecolor}{rgb}{1.000000, 1.000000, 1.000000}
\pgfsetfillcolor{dialinecolor}
\fill (16.536250\du,14.650000\du)--(16.536250\du,16.550000\du)--(21.363750\du,16.550000\du)--(21.363750\du,14.650000\du)--cycle;
\pgfsetlinewidth{0.100000\du}
\pgfsetdash{}{0pt}
\pgfsetdash{}{0pt}
\pgfsetmiterjoin
\definecolor{dialinecolor}{rgb}{0.000000, 0.000000, 0.000000}
\pgfsetstrokecolor{dialinecolor}
\draw (16.536250\du,14.650000\du)--(16.536250\du,16.550000\du)--(21.363750\du,16.550000\du)--(21.363750\du,14.650000\du)--cycle;
% setfont left to latex
\definecolor{dialinecolor}{rgb}{0.000000, 0.000000, 0.000000}
\pgfsetstrokecolor{dialinecolor}
\node at (18.950000\du,15.795000\du){Microphone};
\definecolor{dialinecolor}{rgb}{1.000000, 1.000000, 1.000000}
\pgfsetfillcolor{dialinecolor}
\fill (16.275000\du,5.600000\du)--(16.275000\du,7.550000\du)--(21.625000\du,7.550000\du)--(21.625000\du,5.600000\du)--cycle;
\pgfsetlinewidth{0.100000\du}
\pgfsetdash{}{0pt}
\pgfsetdash{}{0pt}
\pgfsetmiterjoin
\definecolor{dialinecolor}{rgb}{1.000000, 1.000000, 1.000000}
\pgfsetstrokecolor{dialinecolor}
\draw (16.275000\du,5.600000\du)--(16.275000\du,7.550000\du)--(21.625000\du,7.550000\du)--(21.625000\du,5.600000\du)--cycle;
% setfont left to latex
\definecolor{dialinecolor}{rgb}{0.000000, 0.000000, 0.000000}
\pgfsetstrokecolor{dialinecolor}
\node at (18.950000\du,6.770000\du){Sensor struct};
\pgfsetlinewidth{0.100000\du}
\pgfsetdash{}{0pt}
\pgfsetdash{}{0pt}
\pgfsetbuttcap
{
\definecolor{dialinecolor}{rgb}{0.000000, 0.000000, 0.000000}
\pgfsetfillcolor{dialinecolor}
% was here!!!
\pgfsetarrowsend{latex}
\definecolor{dialinecolor}{rgb}{0.000000, 0.000000, 0.000000}
\pgfsetstrokecolor{dialinecolor}
\draw (13.200000\du,6.650000\du)--(16.700000\du,8.900000\du);
}
\pgfsetlinewidth{0.100000\du}
\pgfsetdash{}{0pt}
\pgfsetdash{}{0pt}
\pgfsetbuttcap
{
\definecolor{dialinecolor}{rgb}{0.000000, 0.000000, 0.000000}
\pgfsetfillcolor{dialinecolor}
% was here!!!
\pgfsetarrowsend{latex}
\definecolor{dialinecolor}{rgb}{0.000000, 0.000000, 0.000000}
\pgfsetstrokecolor{dialinecolor}
\draw (13.190000\du,9.000000\du)--(16.605000\du,11.250000\du);
}
\pgfsetlinewidth{0.100000\du}
\pgfsetdash{}{0pt}
\pgfsetdash{}{0pt}
\pgfsetbuttcap
{
\definecolor{dialinecolor}{rgb}{0.000000, 0.000000, 0.000000}
\pgfsetfillcolor{dialinecolor}
% was here!!!
\pgfsetarrowsend{latex}
\definecolor{dialinecolor}{rgb}{0.000000, 0.000000, 0.000000}
\pgfsetstrokecolor{dialinecolor}
\draw (13.492500\du,11.250000\du)--(16.402500\du,13.350000\du);
}
\pgfsetlinewidth{0.100000\du}
\pgfsetdash{}{0pt}
\pgfsetdash{}{0pt}
\pgfsetbuttcap
{
\definecolor{dialinecolor}{rgb}{0.000000, 0.000000, 0.000000}
\pgfsetfillcolor{dialinecolor}
% was here!!!
\pgfsetarrowsend{latex}
\definecolor{dialinecolor}{rgb}{0.000000, 0.000000, 0.000000}
\pgfsetstrokecolor{dialinecolor}
\draw (21.800000\du,11.175000\du)--(23.050000\du,11.175000\du);
}
\pgfsetlinewidth{0.100000\du}
\pgfsetdash{}{0pt}
\pgfsetdash{}{0pt}
\pgfsetbuttcap
{
\definecolor{dialinecolor}{rgb}{0.000000, 0.000000, 0.000000}
\pgfsetfillcolor{dialinecolor}
% was here!!!
\pgfsetarrowsend{latex}
\definecolor{dialinecolor}{rgb}{0.000000, 0.000000, 0.000000}
\pgfsetstrokecolor{dialinecolor}
\draw (27.400000\du,11.175000\du)--(28.600000\du,11.125000\du);
}
\pgfsetlinewidth{0.100000\du}
\pgfsetdash{}{0pt}
\pgfsetdash{}{0pt}
\pgfsetmiterjoin
\pgfsetbuttcap
{
\definecolor{dialinecolor}{rgb}{0.000000, 0.000000, 0.000000}
\pgfsetfillcolor{dialinecolor}
% was here!!!
\pgfsetarrowsend{latex}
{\pgfsetcornersarced{\pgfpoint{0.000000\du}{0.000000\du}}\definecolor{dialinecolor}{rgb}{0.000000, 0.000000, 0.000000}
\pgfsetstrokecolor{dialinecolor}
\draw (8.580949\du,13.500000\du)--(7.350000\du,13.500000\du)--(7.350000\du,16.500000\du)--(8.400000\du,16.500000\du);
}}
\pgfsetlinewidth{0.100000\du}
\pgfsetdash{}{0pt}
\pgfsetdash{}{0pt}
\pgfsetmiterjoin
\pgfsetbuttcap
{
\definecolor{dialinecolor}{rgb}{0.000000, 0.000000, 0.000000}
\pgfsetfillcolor{dialinecolor}
% was here!!!
\pgfsetarrowsend{latex}
{\pgfsetcornersarced{\pgfpoint{0.000000\du}{0.000000\du}}\definecolor{dialinecolor}{rgb}{0.000000, 0.000000, 0.000000}
\pgfsetstrokecolor{dialinecolor}
\draw (13.550000\du,16.500000\du)--(15.043125\du,16.500000\du)--(15.043125\du,15.600000\du)--(16.536250\du,15.600000\du);
}}
\end{tikzpicture}
}
\caption{Diagrama de Flujo de datos}
\label{fig:data_flow}
\end{figure}


\subsubsection{Struct de sensores}
La toma de datos de los sensores es asincrónica respecto al ciclo de control principal. Entonces se
utiliza una estructura para definir la interfaz de comunicación entre ambas partes. Este es el
struct de sensores, Como se muestra en la figura \ref{fig:data_flow}. En este se guarda la
información relevante que necesita la máquina de estados para definir el comportamiento del
sistema. Se guardan 4 valores, el primero es el valor de tiempo. En este se guarda la cantidad de
tiempo que ha pasado, ya que algunas transiciones de estado están definidas por \textit{timeouts},
como por ejemplo cuando la lámpara se enciende cuando la habitación está oscura y se detecta ruido,
el retorno al modo automático está asociado a un \textit{timeout}.

El segundo guarda un booleano que indica si el botón de control ha sido presionado. Entonces cuando
se atiende la interrupción de presionar el botón, se guarda el dato en el struct de sensores, y
en luego la máquina de estados se encarga de retornar el valor a su estado inicial.

La tercer entrada del struct de sensores se refiere al sensor de luz. Cuando se atiende la
interrupción del sensor de luz, se compara con el \textit{threshold} establecido experimentalmente y
se guarda en el struct un booleano que indica si la lectura actual superó o no dicho
threshold. También estaba la opción de tener una atención de interrupción más simple y solo guardar
el dato del sensor en el struct de sensores y que luego el loop de control principal se encargara de
hacer la comparación. Sin embargo se tomó la decisión que el loop principal fuera independiente del
hardware y por tanto debería funcionar independientemente del threshold de luz que se utilice (que
depende del sensor). Entonces era mejor que el código HD se encargara de la comparación.

La última entrada del struct de sensores es para el micrófono, y se guarda un booleano que indica si
el último segundo cumple la condición de estar sobre el promedio establecido por los cinco segundos
anteriores. Se decide guardar solo el booleano y no los datos por la misma razón expuesta para el
caso del sensor de luz.

\subsubsection{Fifo de manejo del micrófono}

Una de las operaciones más complejas de este primer laboratorio era identificar si se cumplían las
condiciones para encender la lámpara. Para el caso del sensor de luz es muy sencillo porque no se
debe tomar en cuenta la historia, sino que solo importan datos inmediatos. Por otra parte para el
caso del micrófono se debe tomar en cuenta los últimos 5 segundos, y compararlo con el último
segundo escuchado.

Para mantener el criterio de diseño de modularizar lo más posible, se decide crear una clase que se
encarga de manejar este problema. A lo interno tiene un fifo donde se guardan los datos del
micrófono y tiene métodos para agregar nuevas muestras y para identificar si se cumple o no la
condición de sonido para encender la lámpara.

Lo primero que se debe considerar es que se tienen dos fifos. El primero guarda los datos
recolectados por los últimos 6 segundos. Esto se hace porque según las especificaciones se debe
comparar el último segundo con el promedio de los últimos 5 segundos. Si solo se guardan 5 segundos
entonces el promedio del último segundo estaría contemplado dentro del promedio de los cinco
segundos, entonces para mejorar la correctitud del sistema se tomó la decisión de guardar un segundo
más, de manera que los datos del último segundo no afecten el promedio contra el que se deben
comparar.

Para optimizar el proceso de obtener el promedio de los cinco segundos, este se
actualiza cada vez que se agrega un nuevo dato. Se sabe el dato entrante (a la parte del fifo que
define los cinco segundos, este no es el dato recién agregado pues pertenece al último segundo, y
no se debe tomar en cuenta para el promedio), y el dato saliente, por lo que se puede utilizar
(\ref{eq:promedio}) para calcular el promedio. Donde $N$ es la cantidad de muestras en los cinco
segundos (depende de la frecuencia de muestreo), $s$ se refiere a una muestra o \textit{sample}, y
$P$ se refiere al promedio.

\begin{equation}
  P_{new} = P_{old}-\frac{s_{out}}{N}+\frac{s_{in}}{N}
\end{equation}

El segundo fifo se utiliza para identificar si se cumple la condición necesaria de intensidad sonora
para encender la lámpara, en este solo se contemplan las muestras del último segundo.  Cada vez que
entra una muestra nueva, se compara con el promedio actual de los cinco segundos, y en caso de ser
mayor al umbral definido, se guarda como un acierto. Cuando todos los valores en el fifo son
aciertos se sabe que se cumple la condición. No se hace una simple comparación de promedios porque
experimentalmente se descubrió que era muy fácil activar la lámpara, lo que aumentaba los falsos
positivos. Luego de agregar este fifo extra se observó un mejor comportamiento.

\section{Máquina de estados}

La máquina de estados es el módulo que toma los valores del struct de sensores y con ellos decide el
estado del sistema. Además hace las llamadas a las funciones de los actuadores, que en este caso son
las lámparas. 

Se definen los siguientes estados:

\begin{itemize}
\item \textbf{None}: Este es el estado con el que se inicializa el objeto. 
\item \textbf{Init}: Indica que el sistema se encuentra en el estado de inicialización, que en el
  caso particular de este proyecto, ocurre mientras se configuran los periféricos.
\item \textbf{Alive\_sequence}: Indica que el sistema se encuentra realizando la secuencia de
  inicialización para indicar su correcto funcionamiento. Hasta que se complete la secuencia, se
  pasa al siguiente estado. 
\item \textbf{ON}: La lámpara se encuentra encendida.
\item \textbf{OFF}: La lámpara se encuentra apagada.
\item \textbf{Manual\_control}: Este es el estado al que entra el sistema cuando se presiona el
  botón de control y la luz estaba apagada, entonces se enciende por un tiempo determinado. Es
  distinto a \textbf{ON} porque no se hacen lecturas de luz o sonido para apagar la lámpara antes
  qeu se complete el tiempo.
\item \textbf{Fail}: Este es el estado para manejo de errores. En este proyecto no hay rutinas que
  puedan generar errores. Lo peor es que haya un error con los sensores, pero en ese caso simplemente
  no se hacen actualizaciones al struct de sensores. 
\item \textbf{Deinit}: Deinicialización del sistema. Como se espera un funcionamiento continuo este
  estado no se utiliza, pero sería utilizado en caso de fallas críticas. 


  
\end{itemize}


\newpage
%-------------------------------
\section{Muestra del Código fuente}
%-------------------------------



\newpage
%-------------------------------
\section{Recomendaciones}
%-------------------------------


\newpage
%----------------------
% Bibliografía
%----------------------


%\bibliographystyle{apacite}
%\bibliography{../general/biblografia.bib}
%---------------------------------------------------------------------------------------------------------------
\end{document}