\documentclass[12pt,letterpaper]{article}
\usepackage[utf8]{inputenc}
\usepackage[T1]{fontenc}
\usepackage[activeacute,spanish]{babel}
\usepackage[left=18mm,right=18mm,top=21mm,bottom=21mm,letterpaper]{geometry}%
\usepackage{helvet}
\usepackage{tikz}
\usepackage{amsmath,amsfonts,amssymb,commath}
\usepackage{graphicx}
\usepackage{color}
\usepackage{xcolor}
\usepackage{verbatim}
\usepackage{tabls}
\usepackage[space]{grffile}
\usepackage{url}
\usepackage{listings}
%\usepackage{circuitikz}
\usepackage{siunitx}
\usepackage{fancyhdr}   
\pagestyle{fancy}
\usepackage{multicol,multirow}
\usepackage{textcomp}
\usepackage{booktabs}
\usepackage[colorlinks=true,urlcolor=blue,linkcolor=black,citecolor=black]{hyperref} 
\usepackage{pdfpages}   %incluir paginas de pdf externo, para los anexos
\usepackage{appendix}
\usepackage{caption}
\usepackage{subcaption}  
\usepackage{apacite}
\usepackage{natbib}
\usepackage{rotating}
\usepackage[section]{placeins}

\lhead{ Laboratorio Sistemas Incrustados}
\chead{}
\rhead{Laboratorio 1}   % Aquí va el numero de experimento, al igual que en el titulo
\lfoot{Escuela de Ingeniería Eléctrica}
\cfoot{\thepage}
\rfoot{Universidad de Costa Rica}

\definecolor{vgreen}{RGB}{104,180,104}
\definecolor{vblue}{RGB}{49,49,255}
\definecolor{vorange}{RGB}{255,143,102}

\lstset{
  language=Verilog,
  basicstyle=\small,
  keywordstyle=\color{vblue},
  identifierstyle=\color{black},
  commentstyle=\color{vgreen},
  numbers=left,
  numberstyle={\tiny \color{black}},
  numbersep=10pt,
  tabsize=4
}

%-------------------------------------------------------------

%\renewcommand{\labelenumi}{\alph{enumi}.}
%\addto\captionsspanish{\renewcommand{\tablename}{Tabla}}                    % Cambiar nombre a tablas
%\addto\captionsspanish{\renewcommand{\listtablename}{Índice de tablas}}     % Cambiar nombre a lista de tablas
\pagenumbering{Roman}
%------------------------------------------------------------

\author{ Daniel García Vaglio, B42781\\ Esteban Zamora Alvarado, B47769 \\
\\ {\small Grupo 01}\\ Profesor: Esteban Ortiz \vspace*{3.0in}}

\begin{document}
%Header-Make sure you update this information!!!!
\noindent
\large\textbf{Laboratorio III: reporte} \hfill Laboratorio de sistemas Incrustados \\
\normalsize Esteban Zamora Alvarado \hfill Carné: B47769 \\
Daniel Esteban García Vaglio \hfill Carné: B42781 \\
Profesor: Esteban Ortiz  \hfill Fecha: 20-Setiembre-2017 \\

\section{Problema a Resolver}
\label{sec:Problem}
Se debe diseñar e implementar un sistema de radio para automóviles. Esto implica realizar la
interfaz gráfica, y una interfaz de audio. Para Poder realizar la prueba de concepto del sistema, es
necesario generar una imagen de linux embebido para ARM, y ejecutarlo con QEMU. 

\section{Contexto de Mercado}
\label{sec:Context}
Existen varias opciones en el mercado para radios de autos, en particular destacan los que utilizan
android. Estos utilizan un sistema operativo ya conocido para la mayoría de los usuarios como lo es
Android y lo incorporan al radio del carro. Estos radios son fáciles de usar y utilizan un sistema
muy robusto y probado. Los precios tiene mucha variación, van desde 8 mil hasta medio millón de
colones.

Otra alternativa, es utilizar un radio simple sin pantalla, pero con bluetooth. Ahora que los
celulares son cada vez más comunes, se está apostando por quitar la interfaz gráfica del radio del
carro y proporcionar una App para controlar el radio desde el celular.  


\section{Solución Propuesta}
\label{sec:solution}
Para la interfaz gráfica se propone utilizar Qt. Esta tiene un conjunto de bibliotecas que permite
el desarrollo rápido de interfaces gráficas de buen aspecto. Además ya existen recetas para YOCTO
que facilitan el desarrollo para sistemas embebidos. Para la interfaz de audio se utiliza Gstreamer
por razones parecidas.

El sistema está diseñado para ser lo más modular posible. de manera que se puedan hacer cambios en
los componentes sin que se afecte al sistema completo. De esta manera la interfaz gráfica y la
interfaz de audio se ejecutan por aparte, cada uno con su propio thread principal. Se utilizan
mensajes de coordinación entre ambos, que definen la interfaz de interconexión.

La interfaz de audio es una máquina de estados (que utiliza los mismos estados definidos para el
pipeline de GStreamer). Y se utiliza un pipeline simple para poder tomar archivos mp3 de memoria,
decodificarlos, y reproducirlos.

La interfaz gráfica se divide en clases. Una clase para cada pantalla: inicio, radio, mp3. De igual
manera se implementaron de la manera más independiente posible. 

%--------------------------------
\section{Implementación}
\label{sec:implementation}
%-------------------------------
\subsection{Interfaz Gráfica (Qt)}
\label{sec:qt}

A pesar que Qt proporciona herramientas para facilitar el desarrollo y configuración de interfaces
gráficas como Qt creator, y QML, estas no se utilizaron. Esta medida se tomó para facilitar la parte
de incorporar la aplicación a Yocto. Entonces la aplicación completa se puede compilar como un
programa normal de C++ (linkeado a bibliotecas de Qt5). En general se evitan las palabras en los
botones. La gran mayoría utilizan íconos, de esta manera la interfaz es más intuitiva para el
usuario. Estos se consiguieron de una página que los ofrece de forma libre. %FIXME: add link

Se definen 4 clases. La primera es un contenedor que se encarga de definir la interfaz entre las
pantallas con las que interactúa el usuario. En esta se tiene los Slots que manejan la transición
entre las distintas pantallas, a saber: bienvenida, radio, mp3.

La pantalla de bienvenida es la más simple. Esta consta de dos botones. El primero despliega la
pantalla de control de radio, y el otro despliega la pantalla de control de Mp3. La pantalla de
radio es un poco más compleja. En esta se tiene el texto principal que despliega la emisora
seleccionada, un selector de AM y FM, botones para seleccionar la emisora y un botón para devolverse
a la pantalla de inicio. Se mantienen dos contadores, uno para FM y el otro para AM. Cuando se
presiona los botones de búsqueda se incrementa o decrementa apropiadamente el contador, y se
actualiza el valor desplegado en el texto principal.

La pantalla de Mp3 es la más compleja. En esta se tiene también un texto principal, donde se
despliega el nombre de la canción seleccionada. Se tienen los botones para seleccionar la canción y
el botón de play-pausa, y el de stop. La lógica para seleccionar la canción es muy parecido al de
selección de emisora, solo que en este caso el contador es el índice de un array con los nombres de
las canciones. Los botones en esta pantalla además de actualizar la visualización de la interfaz
también envían mensajes al thread de audio. En este mensaje se envía la acción que se debe realizar
(play, pause, stop, cambiar de canción), de manera que el audio se sincronice con la interfaz
gráfica. 

\subsection{Interfaz de Audio (GStreamer)}
\label{sec:gstramer}
Las canciones para esta parte se tomaron de una fuente que ofrece música de licencia libre. %Fixme
                                %add link


Para la interfaz de Audio se tiene un thread por aparte. Este es un ciclo de control con una
frecuencia determinada que se encarga estar escuchando constantemente los mensajes de la interfaz
gráfica para modificar su estado apropiadamente. Cada uno de los botones de la pantalla de Mp3 envía
el comando que corresponde a su acción. El de play hace que el pipeline comience a ejecutarse, el se
pause provoca la pausa. El botón de stop provoca que se detenga la ejecución y que se retorne al
inicio de la canción. Los botones de búsqueda, hacia adelante y hacia atrás, envían el comando de
incrementar o decrementar el índice de la canción. Para lograrlo primero se hace Stop, y luego se
carga el nuevo archivo en al fuente del pipeline. 

El pipeline consta de 5 etapas. La primera el se source (filesrc), que toma el archivo y lo alista
para el resto del Pipeline. Luego se tiene el parser (mpegaudioparse), que hace un preprocesamiento
que se necesita por trabajar con MP3. La tercera etapa es el decodificador (mpg123audiodec), y este
se conecta con el converidor (audioconvert). Para este punto el stream está listo para la interfaz
de audio, entonces se utiliza pulsesink para conectarlo con pulseaudio. 


\subsection{Yocto}
\label{sec:yocto}


%-------------------------------
\section{Problemas encontrados y posibles mejoras}
\label{sec:possible_fixes}
% -------------------------------



%
%\begin{figure}
%  \centering
%\scalebox{.75}{% Graphic for TeX using PGF
% Title: /home/daniel/repos/ucr/embeded/Lab1/doc/data_flow.dia
% Creator: Dia v0.97.3
% CreationDate: Wed Sep 20 13:35:17 2017
% For: daniel
% \usepackage{tikz}
% The following commands are not supported in PSTricks at present
% We define them conditionally, so when they are implemented,
% this pgf file will use them.
\ifx\du\undefined
  \newlength{\du}
\fi
\setlength{\du}{15\unitlength}
\begin{tikzpicture}
\pgftransformxscale{1.000000}
\pgftransformyscale{-1.000000}
\definecolor{dialinecolor}{rgb}{0.000000, 0.000000, 0.000000}
\pgfsetstrokecolor{dialinecolor}
\definecolor{dialinecolor}{rgb}{1.000000, 1.000000, 1.000000}
\pgfsetfillcolor{dialinecolor}
\definecolor{dialinecolor}{rgb}{1.000000, 1.000000, 1.000000}
\pgfsetfillcolor{dialinecolor}
\fill (8.695000\du,5.700000\du)--(8.695000\du,7.600000\du)--(13.200000\du,7.600000\du)--(13.200000\du,5.700000\du)--cycle;
\pgfsetlinewidth{0.100000\du}
\pgfsetdash{}{0pt}
\pgfsetdash{}{0pt}
\pgfsetmiterjoin
\definecolor{dialinecolor}{rgb}{0.000000, 0.000000, 0.000000}
\pgfsetstrokecolor{dialinecolor}
\draw (8.695000\du,5.700000\du)--(8.695000\du,7.600000\du)--(13.200000\du,7.600000\du)--(13.200000\du,5.700000\du)--cycle;
% setfont left to latex
\definecolor{dialinecolor}{rgb}{0.000000, 0.000000, 0.000000}
\pgfsetstrokecolor{dialinecolor}
\node at (10.947500\du,6.845000\du){Time};
\definecolor{dialinecolor}{rgb}{1.000000, 1.000000, 1.000000}
\pgfsetfillcolor{dialinecolor}
\fill (8.397500\du,10.300000\du)--(8.397500\du,12.200000\du)--(13.492500\du,12.200000\du)--(13.492500\du,10.300000\du)--cycle;
\pgfsetlinewidth{0.100000\du}
\pgfsetdash{}{0pt}
\pgfsetdash{}{0pt}
\pgfsetmiterjoin
\definecolor{dialinecolor}{rgb}{0.000000, 0.000000, 0.000000}
\pgfsetstrokecolor{dialinecolor}
\draw (8.397500\du,10.300000\du)--(8.397500\du,12.200000\du)--(13.492500\du,12.200000\du)--(13.492500\du,10.300000\du)--cycle;
% setfont left to latex
\definecolor{dialinecolor}{rgb}{0.000000, 0.000000, 0.000000}
\pgfsetstrokecolor{dialinecolor}
\node at (10.945000\du,11.445000\du){Light Sensor};
\definecolor{dialinecolor}{rgb}{1.000000, 1.000000, 1.000000}
\pgfsetfillcolor{dialinecolor}
\fill (8.631250\du,12.550000\du)--(8.631250\du,14.450000\du)--(13.458750\du,14.450000\du)--(13.458750\du,12.550000\du)--cycle;
\pgfsetlinewidth{0.100000\du}
\pgfsetdash{}{0pt}
\pgfsetdash{}{0pt}
\pgfsetmiterjoin
\definecolor{dialinecolor}{rgb}{0.000000, 0.000000, 0.000000}
\pgfsetstrokecolor{dialinecolor}
\draw (8.631250\du,12.550000\du)--(8.631250\du,14.450000\du)--(13.458750\du,14.450000\du)--(13.458750\du,12.550000\du)--cycle;
% setfont left to latex
\definecolor{dialinecolor}{rgb}{0.000000, 0.000000, 0.000000}
\pgfsetstrokecolor{dialinecolor}
\node at (11.045000\du,13.695000\du){Microphone};
\definecolor{dialinecolor}{rgb}{1.000000, 1.000000, 1.000000}
\pgfsetfillcolor{dialinecolor}
\fill (8.500000\du,8.050000\du)--(8.500000\du,9.950000\du)--(13.190000\du,9.950000\du)--(13.190000\du,8.050000\du)--cycle;
\pgfsetlinewidth{0.100000\du}
\pgfsetdash{}{0pt}
\pgfsetdash{}{0pt}
\pgfsetmiterjoin
\definecolor{dialinecolor}{rgb}{0.000000, 0.000000, 0.000000}
\pgfsetstrokecolor{dialinecolor}
\draw (8.500000\du,8.050000\du)--(8.500000\du,9.950000\du)--(13.190000\du,9.950000\du)--(13.190000\du,8.050000\du)--cycle;
% setfont left to latex
\definecolor{dialinecolor}{rgb}{0.000000, 0.000000, 0.000000}
\pgfsetstrokecolor{dialinecolor}
\node at (10.845000\du,9.195000\du){Button};
\definecolor{dialinecolor}{rgb}{1.000000, 1.000000, 1.000000}
\pgfsetfillcolor{dialinecolor}
\fill (8.400000\du,14.850000\du)--(8.400000\du,18.150000\du)--(13.550000\du,18.150000\du)--(13.550000\du,14.850000\du)--cycle;
\pgfsetlinewidth{0.100000\du}
\pgfsetdash{}{0pt}
\pgfsetdash{}{0pt}
\pgfsetmiterjoin
\definecolor{dialinecolor}{rgb}{0.000000, 0.000000, 0.000000}
\pgfsetstrokecolor{dialinecolor}
\draw (8.400000\du,14.850000\du)--(8.400000\du,18.150000\du)--(13.550000\du,18.150000\du)--(13.550000\du,14.850000\du)--cycle;
% setfont left to latex
\definecolor{dialinecolor}{rgb}{0.000000, 0.000000, 0.000000}
\pgfsetstrokecolor{dialinecolor}
\node at (10.975000\du,16.295000\du){Microphone};
% setfont left to latex
\definecolor{dialinecolor}{rgb}{0.000000, 0.000000, 0.000000}
\pgfsetstrokecolor{dialinecolor}
\node at (10.975000\du,17.095000\du){FIFO};
\definecolor{dialinecolor}{rgb}{1.000000, 1.000000, 1.000000}
\pgfsetfillcolor{dialinecolor}
\fill (23.050000\du,9.550000\du)--(23.050000\du,12.800000\du)--(27.400000\du,12.800000\du)--(27.400000\du,9.550000\du)--cycle;
\pgfsetlinewidth{0.100000\du}
\pgfsetdash{}{0pt}
\pgfsetdash{}{0pt}
\pgfsetmiterjoin
\definecolor{dialinecolor}{rgb}{0.000000, 0.000000, 0.000000}
\pgfsetstrokecolor{dialinecolor}
\draw (23.050000\du,9.550000\du)--(23.050000\du,12.800000\du)--(27.400000\du,12.800000\du)--(27.400000\du,9.550000\du)--cycle;
% setfont left to latex
\definecolor{dialinecolor}{rgb}{0.000000, 0.000000, 0.000000}
\pgfsetstrokecolor{dialinecolor}
\node at (25.225000\du,10.970000\du){State};
% setfont left to latex
\definecolor{dialinecolor}{rgb}{0.000000, 0.000000, 0.000000}
\pgfsetstrokecolor{dialinecolor}
\node at (25.225000\du,11.770000\du){Machine};
\definecolor{dialinecolor}{rgb}{1.000000, 1.000000, 1.000000}
\pgfsetfillcolor{dialinecolor}
\fill (28.600000\du,10.175000\du)--(28.600000\du,12.075000\du)--(31.812500\du,12.075000\du)--(31.812500\du,10.175000\du)--cycle;
\pgfsetlinewidth{0.100000\du}
\pgfsetdash{}{0pt}
\pgfsetdash{}{0pt}
\pgfsetmiterjoin
\definecolor{dialinecolor}{rgb}{0.000000, 0.000000, 0.000000}
\pgfsetstrokecolor{dialinecolor}
\draw (28.600000\du,10.175000\du)--(28.600000\du,12.075000\du)--(31.812500\du,12.075000\du)--(31.812500\du,10.175000\du)--cycle;
% setfont left to latex
\definecolor{dialinecolor}{rgb}{0.000000, 0.000000, 0.000000}
\pgfsetstrokecolor{dialinecolor}
\node at (30.206250\du,11.320000\du){Lamps};
\definecolor{dialinecolor}{rgb}{1.000000, 1.000000, 1.000000}
\pgfsetfillcolor{dialinecolor}
\fill (16.100000\du,5.500000\du)--(16.100000\du,16.850000\du)--(21.800000\du,16.850000\du)--(21.800000\du,5.500000\du)--cycle;
\pgfsetlinewidth{0.100000\du}
\pgfsetdash{}{0pt}
\pgfsetdash{}{0pt}
\pgfsetmiterjoin
\definecolor{dialinecolor}{rgb}{0.000000, 0.000000, 0.000000}
\pgfsetstrokecolor{dialinecolor}
\draw (16.100000\du,5.500000\du)--(16.100000\du,16.850000\du)--(21.800000\du,16.850000\du)--(21.800000\du,5.500000\du)--cycle;
% setfont left to latex
\definecolor{dialinecolor}{rgb}{0.000000, 0.000000, 0.000000}
\pgfsetstrokecolor{dialinecolor}
\node at (18.950000\du,11.370000\du){};
\definecolor{dialinecolor}{rgb}{1.000000, 1.000000, 1.000000}
\pgfsetfillcolor{dialinecolor}
\fill (16.700000\du,7.950000\du)--(16.700000\du,9.850000\du)--(21.200000\du,9.850000\du)--(21.200000\du,7.950000\du)--cycle;
\pgfsetlinewidth{0.100000\du}
\pgfsetdash{}{0pt}
\pgfsetdash{}{0pt}
\pgfsetmiterjoin
\definecolor{dialinecolor}{rgb}{0.000000, 0.000000, 0.000000}
\pgfsetstrokecolor{dialinecolor}
\draw (16.700000\du,7.950000\du)--(16.700000\du,9.850000\du)--(21.200000\du,9.850000\du)--(21.200000\du,7.950000\du)--cycle;
% setfont left to latex
\definecolor{dialinecolor}{rgb}{0.000000, 0.000000, 0.000000}
\pgfsetstrokecolor{dialinecolor}
\node at (18.950000\du,9.095000\du){Time};
\definecolor{dialinecolor}{rgb}{1.000000, 1.000000, 1.000000}
\pgfsetfillcolor{dialinecolor}
\fill (16.605000\du,10.300000\du)--(16.605000\du,12.200000\du)--(21.295000\du,12.200000\du)--(21.295000\du,10.300000\du)--cycle;
\pgfsetlinewidth{0.100000\du}
\pgfsetdash{}{0pt}
\pgfsetdash{}{0pt}
\pgfsetmiterjoin
\definecolor{dialinecolor}{rgb}{0.000000, 0.000000, 0.000000}
\pgfsetstrokecolor{dialinecolor}
\draw (16.605000\du,10.300000\du)--(16.605000\du,12.200000\du)--(21.295000\du,12.200000\du)--(21.295000\du,10.300000\du)--cycle;
% setfont left to latex
\definecolor{dialinecolor}{rgb}{0.000000, 0.000000, 0.000000}
\pgfsetstrokecolor{dialinecolor}
\node at (18.950000\du,11.445000\du){Button};
\definecolor{dialinecolor}{rgb}{1.000000, 1.000000, 1.000000}
\pgfsetfillcolor{dialinecolor}
\fill (16.402500\du,12.400000\du)--(16.402500\du,14.300000\du)--(21.497500\du,14.300000\du)--(21.497500\du,12.400000\du)--cycle;
\pgfsetlinewidth{0.100000\du}
\pgfsetdash{}{0pt}
\pgfsetdash{}{0pt}
\pgfsetmiterjoin
\definecolor{dialinecolor}{rgb}{0.000000, 0.000000, 0.000000}
\pgfsetstrokecolor{dialinecolor}
\draw (16.402500\du,12.400000\du)--(16.402500\du,14.300000\du)--(21.497500\du,14.300000\du)--(21.497500\du,12.400000\du)--cycle;
% setfont left to latex
\definecolor{dialinecolor}{rgb}{0.000000, 0.000000, 0.000000}
\pgfsetstrokecolor{dialinecolor}
\node at (18.950000\du,13.545000\du){Light Sensor};
\definecolor{dialinecolor}{rgb}{1.000000, 1.000000, 1.000000}
\pgfsetfillcolor{dialinecolor}
\fill (16.536250\du,14.650000\du)--(16.536250\du,16.550000\du)--(21.363750\du,16.550000\du)--(21.363750\du,14.650000\du)--cycle;
\pgfsetlinewidth{0.100000\du}
\pgfsetdash{}{0pt}
\pgfsetdash{}{0pt}
\pgfsetmiterjoin
\definecolor{dialinecolor}{rgb}{0.000000, 0.000000, 0.000000}
\pgfsetstrokecolor{dialinecolor}
\draw (16.536250\du,14.650000\du)--(16.536250\du,16.550000\du)--(21.363750\du,16.550000\du)--(21.363750\du,14.650000\du)--cycle;
% setfont left to latex
\definecolor{dialinecolor}{rgb}{0.000000, 0.000000, 0.000000}
\pgfsetstrokecolor{dialinecolor}
\node at (18.950000\du,15.795000\du){Microphone};
\definecolor{dialinecolor}{rgb}{1.000000, 1.000000, 1.000000}
\pgfsetfillcolor{dialinecolor}
\fill (16.275000\du,5.600000\du)--(16.275000\du,7.550000\du)--(21.625000\du,7.550000\du)--(21.625000\du,5.600000\du)--cycle;
\pgfsetlinewidth{0.100000\du}
\pgfsetdash{}{0pt}
\pgfsetdash{}{0pt}
\pgfsetmiterjoin
\definecolor{dialinecolor}{rgb}{1.000000, 1.000000, 1.000000}
\pgfsetstrokecolor{dialinecolor}
\draw (16.275000\du,5.600000\du)--(16.275000\du,7.550000\du)--(21.625000\du,7.550000\du)--(21.625000\du,5.600000\du)--cycle;
% setfont left to latex
\definecolor{dialinecolor}{rgb}{0.000000, 0.000000, 0.000000}
\pgfsetstrokecolor{dialinecolor}
\node at (18.950000\du,6.770000\du){Sensor struct};
\pgfsetlinewidth{0.100000\du}
\pgfsetdash{}{0pt}
\pgfsetdash{}{0pt}
\pgfsetbuttcap
{
\definecolor{dialinecolor}{rgb}{0.000000, 0.000000, 0.000000}
\pgfsetfillcolor{dialinecolor}
% was here!!!
\pgfsetarrowsend{latex}
\definecolor{dialinecolor}{rgb}{0.000000, 0.000000, 0.000000}
\pgfsetstrokecolor{dialinecolor}
\draw (13.200000\du,6.650000\du)--(16.700000\du,8.900000\du);
}
\pgfsetlinewidth{0.100000\du}
\pgfsetdash{}{0pt}
\pgfsetdash{}{0pt}
\pgfsetbuttcap
{
\definecolor{dialinecolor}{rgb}{0.000000, 0.000000, 0.000000}
\pgfsetfillcolor{dialinecolor}
% was here!!!
\pgfsetarrowsend{latex}
\definecolor{dialinecolor}{rgb}{0.000000, 0.000000, 0.000000}
\pgfsetstrokecolor{dialinecolor}
\draw (13.190000\du,9.000000\du)--(16.605000\du,11.250000\du);
}
\pgfsetlinewidth{0.100000\du}
\pgfsetdash{}{0pt}
\pgfsetdash{}{0pt}
\pgfsetbuttcap
{
\definecolor{dialinecolor}{rgb}{0.000000, 0.000000, 0.000000}
\pgfsetfillcolor{dialinecolor}
% was here!!!
\pgfsetarrowsend{latex}
\definecolor{dialinecolor}{rgb}{0.000000, 0.000000, 0.000000}
\pgfsetstrokecolor{dialinecolor}
\draw (13.492500\du,11.250000\du)--(16.402500\du,13.350000\du);
}
\pgfsetlinewidth{0.100000\du}
\pgfsetdash{}{0pt}
\pgfsetdash{}{0pt}
\pgfsetbuttcap
{
\definecolor{dialinecolor}{rgb}{0.000000, 0.000000, 0.000000}
\pgfsetfillcolor{dialinecolor}
% was here!!!
\pgfsetarrowsend{latex}
\definecolor{dialinecolor}{rgb}{0.000000, 0.000000, 0.000000}
\pgfsetstrokecolor{dialinecolor}
\draw (21.800000\du,11.175000\du)--(23.050000\du,11.175000\du);
}
\pgfsetlinewidth{0.100000\du}
\pgfsetdash{}{0pt}
\pgfsetdash{}{0pt}
\pgfsetbuttcap
{
\definecolor{dialinecolor}{rgb}{0.000000, 0.000000, 0.000000}
\pgfsetfillcolor{dialinecolor}
% was here!!!
\pgfsetarrowsend{latex}
\definecolor{dialinecolor}{rgb}{0.000000, 0.000000, 0.000000}
\pgfsetstrokecolor{dialinecolor}
\draw (27.400000\du,11.175000\du)--(28.600000\du,11.125000\du);
}
\pgfsetlinewidth{0.100000\du}
\pgfsetdash{}{0pt}
\pgfsetdash{}{0pt}
\pgfsetmiterjoin
\pgfsetbuttcap
{
\definecolor{dialinecolor}{rgb}{0.000000, 0.000000, 0.000000}
\pgfsetfillcolor{dialinecolor}
% was here!!!
\pgfsetarrowsend{latex}
{\pgfsetcornersarced{\pgfpoint{0.000000\du}{0.000000\du}}\definecolor{dialinecolor}{rgb}{0.000000, 0.000000, 0.000000}
\pgfsetstrokecolor{dialinecolor}
\draw (8.580949\du,13.500000\du)--(7.350000\du,13.500000\du)--(7.350000\du,16.500000\du)--(8.400000\du,16.500000\du);
}}
\pgfsetlinewidth{0.100000\du}
\pgfsetdash{}{0pt}
\pgfsetdash{}{0pt}
\pgfsetmiterjoin
\pgfsetbuttcap
{
\definecolor{dialinecolor}{rgb}{0.000000, 0.000000, 0.000000}
\pgfsetfillcolor{dialinecolor}
% was here!!!
\pgfsetarrowsend{latex}
{\pgfsetcornersarced{\pgfpoint{0.000000\du}{0.000000\du}}\definecolor{dialinecolor}{rgb}{0.000000, 0.000000, 0.000000}
\pgfsetstrokecolor{dialinecolor}
\draw (13.550000\du,16.500000\du)--(15.043125\du,16.500000\du)--(15.043125\du,15.600000\du)--(16.536250\du,15.600000\du);
}}
\end{tikzpicture}
}
%\caption{Diagrama de Flujo de datos}
%\label{fig:data_flow}
%\end{figure}


%----------------------
\textbf{ Bibliografía}
% ----------------------

Free Icons. Icon8. https://icons8.com/

Free Music Archive. http://freemusicarchive.org/
%\bibliographystyle{apacite}
%\bibliography{../general/biblografia.bib}
%---------------------------------------------------------------------------------------------------------------
\end{document}