\documentclass[12pt,letterpaper]{article}
\usepackage[utf8]{inputenc}
\usepackage[T1]{fontenc}
\usepackage[activeacute,spanish]{babel}
\usepackage[left=18mm,right=18mm,top=21mm,bottom=21mm,letterpaper]{geometry}%
\usepackage{helvet}
\usepackage{tikz}
\usepackage{amsmath,amsfonts,amssymb,commath}
\usepackage{graphicx}
\usepackage{color}
\usepackage{xcolor}
\usepackage{verbatim}
\usepackage{tabls}
\usepackage[space]{grffile}
\usepackage{url}
\usepackage{listings}
%\usepackage{circuitikz}
\usepackage{siunitx}
\usepackage{fancyhdr}   
\pagestyle{fancy}
\usepackage{multicol,multirow}
\usepackage{textcomp}
\usepackage{booktabs}
\usepackage[colorlinks=true,urlcolor=blue,linkcolor=black,citecolor=black]{hyperref} 
\usepackage{pdfpages}   %incluir paginas de pdf externo, para los anexos
\usepackage{appendix}
\usepackage{caption}
\usepackage{subcaption}  
\usepackage{apacite}
\usepackage{natbib}
\usepackage{rotating}
\usepackage[section]{placeins}

\lhead{ Laboratorio Sistemas Incrustados}
\chead{}
\rhead{Laboratorio 1}   % Aquí va el numero de experimento, al igual que en el titulo
\lfoot{Escuela de Ingeniería Eléctrica}
\cfoot{\thepage}
\rfoot{Universidad de Costa Rica}

\definecolor{vgreen}{RGB}{104,180,104}
\definecolor{vblue}{RGB}{49,49,255}
\definecolor{vorange}{RGB}{255,143,102}

\lstset{
  language=Verilog,
  basicstyle=\small,
  keywordstyle=\color{vblue},
  identifierstyle=\color{black},
  commentstyle=\color{vgreen},
  numbers=left,
  numberstyle={\tiny \color{black}},
  numbersep=10pt,
  tabsize=4
}

%-------------------------------------------------------------

%\renewcommand{\labelenumi}{\alph{enumi}.}
%\addto\captionsspanish{\renewcommand{\tablename}{Tabla}}                    % Cambiar nombre a tablas
%\addto\captionsspanish{\renewcommand{\listtablename}{Índice de tablas}}     % Cambiar nombre a lista de tablas
\pagenumbering{Roman}
%------------------------------------------------------------

\author{ Daniel García Vaglio, B42781\\ Esteban Zamora Alvarado, B47769 \\
\\ {\small Grupo 01}\\ Profesor: Esteban Ortiz \vspace*{3.0in}}

\begin{document}
%Header-Make sure you update this information!!!!
\noindent
\large\textbf{Laboratorio III: reporte} \hfill Laboratorio de sistemas Incrustados \\
\normalsize Esteban Zamora Alvarado \hfill Carné: B47769 \\
Daniel Esteban García Vaglio \hfill Carné: B42781 \\
Profesor: Esteban Ortiz  \hfill Fecha: 20-Setiembre-2017 \\

\section{Problema a Resolver}
\label{sec:Problem}
Se debe diseñar e implementar un sistema de radio para automóviles. Esto implica realizar la
interfaz gráfica, y una interfaz de audio. Para Poder realizar la prueba de concepto del sistema, es
necesario generar una imagen de linux embebido para ARM, y ejecutarlo con QEMU. 

\section{Contexto de Mercado}
\label{sec:Context}

\section{Solución Propuesta}
\label{sec:solution}
Para la interfaz gráfica se propone utilizar Qt. Esta tiene un conjunto de bibliotecas que permite
el desarrollo rápido de interfaces gráficas de buen aspecto. Además ya existen recetas para YOCTO
que facilitan el desarrollo para sistemas embebidos. Para la interfaz de audio se utiliza Gstreamer
por razones parecidas.

El sistema está diseñado para ser lo más modular posible. de manera que se puedan hacer cambios en
los componentes sin que se afecte al sistema completo. De esta manera la interfaz gráfica y la
interfaz de audio se ejecutan por aparte, cada uno con su propio thread principal. Se utilizan
mensajes de coordinación entre ambos, que definen la interfaz de interconexión.

La interfaz de audio es una máquina de estados (que utiliza los mismos estados definidos para el
pipeline de GStreamer). Y se utiliza un pipeline simple para poder tomar archivos mp3 de memoria,
decodificarlos, y reproducirlos.

La interfaz gráfica se divide en clases. Una clase para cada pantalla: inicio, radio, mp3. De igual
manera se implementaron de la manera más independiente posible. 

%--------------------------------
\section{Implementación}
\label{sec:implementation}

%-------------------------------
\subsection{Interfaz Gráfica (Qt)}
\label{sec:qt}

\subsection{Interfaz de Audio (GStreamer)}
\label{sec:gstramer}



\subsection{Yocto}
\label{sec:yocto}


%-------------------------------
\section{Problemas encontrados y posibles mejoras}
\label{sec:possible_fixes}
% -------------------------------



%
%\begin{figure}
%  \centering
%\scalebox{.75}{% Graphic for TeX using PGF
% Title: /home/daniel/repos/ucr/embeded/Lab1/doc/data_flow.dia
% Creator: Dia v0.97.3
% CreationDate: Wed Sep 20 13:35:17 2017
% For: daniel
% \usepackage{tikz}
% The following commands are not supported in PSTricks at present
% We define them conditionally, so when they are implemented,
% this pgf file will use them.
\ifx\du\undefined
  \newlength{\du}
\fi
\setlength{\du}{15\unitlength}
\begin{tikzpicture}
\pgftransformxscale{1.000000}
\pgftransformyscale{-1.000000}
\definecolor{dialinecolor}{rgb}{0.000000, 0.000000, 0.000000}
\pgfsetstrokecolor{dialinecolor}
\definecolor{dialinecolor}{rgb}{1.000000, 1.000000, 1.000000}
\pgfsetfillcolor{dialinecolor}
\definecolor{dialinecolor}{rgb}{1.000000, 1.000000, 1.000000}
\pgfsetfillcolor{dialinecolor}
\fill (8.695000\du,5.700000\du)--(8.695000\du,7.600000\du)--(13.200000\du,7.600000\du)--(13.200000\du,5.700000\du)--cycle;
\pgfsetlinewidth{0.100000\du}
\pgfsetdash{}{0pt}
\pgfsetdash{}{0pt}
\pgfsetmiterjoin
\definecolor{dialinecolor}{rgb}{0.000000, 0.000000, 0.000000}
\pgfsetstrokecolor{dialinecolor}
\draw (8.695000\du,5.700000\du)--(8.695000\du,7.600000\du)--(13.200000\du,7.600000\du)--(13.200000\du,5.700000\du)--cycle;
% setfont left to latex
\definecolor{dialinecolor}{rgb}{0.000000, 0.000000, 0.000000}
\pgfsetstrokecolor{dialinecolor}
\node at (10.947500\du,6.845000\du){Time};
\definecolor{dialinecolor}{rgb}{1.000000, 1.000000, 1.000000}
\pgfsetfillcolor{dialinecolor}
\fill (8.397500\du,10.300000\du)--(8.397500\du,12.200000\du)--(13.492500\du,12.200000\du)--(13.492500\du,10.300000\du)--cycle;
\pgfsetlinewidth{0.100000\du}
\pgfsetdash{}{0pt}
\pgfsetdash{}{0pt}
\pgfsetmiterjoin
\definecolor{dialinecolor}{rgb}{0.000000, 0.000000, 0.000000}
\pgfsetstrokecolor{dialinecolor}
\draw (8.397500\du,10.300000\du)--(8.397500\du,12.200000\du)--(13.492500\du,12.200000\du)--(13.492500\du,10.300000\du)--cycle;
% setfont left to latex
\definecolor{dialinecolor}{rgb}{0.000000, 0.000000, 0.000000}
\pgfsetstrokecolor{dialinecolor}
\node at (10.945000\du,11.445000\du){Light Sensor};
\definecolor{dialinecolor}{rgb}{1.000000, 1.000000, 1.000000}
\pgfsetfillcolor{dialinecolor}
\fill (8.631250\du,12.550000\du)--(8.631250\du,14.450000\du)--(13.458750\du,14.450000\du)--(13.458750\du,12.550000\du)--cycle;
\pgfsetlinewidth{0.100000\du}
\pgfsetdash{}{0pt}
\pgfsetdash{}{0pt}
\pgfsetmiterjoin
\definecolor{dialinecolor}{rgb}{0.000000, 0.000000, 0.000000}
\pgfsetstrokecolor{dialinecolor}
\draw (8.631250\du,12.550000\du)--(8.631250\du,14.450000\du)--(13.458750\du,14.450000\du)--(13.458750\du,12.550000\du)--cycle;
% setfont left to latex
\definecolor{dialinecolor}{rgb}{0.000000, 0.000000, 0.000000}
\pgfsetstrokecolor{dialinecolor}
\node at (11.045000\du,13.695000\du){Microphone};
\definecolor{dialinecolor}{rgb}{1.000000, 1.000000, 1.000000}
\pgfsetfillcolor{dialinecolor}
\fill (8.500000\du,8.050000\du)--(8.500000\du,9.950000\du)--(13.190000\du,9.950000\du)--(13.190000\du,8.050000\du)--cycle;
\pgfsetlinewidth{0.100000\du}
\pgfsetdash{}{0pt}
\pgfsetdash{}{0pt}
\pgfsetmiterjoin
\definecolor{dialinecolor}{rgb}{0.000000, 0.000000, 0.000000}
\pgfsetstrokecolor{dialinecolor}
\draw (8.500000\du,8.050000\du)--(8.500000\du,9.950000\du)--(13.190000\du,9.950000\du)--(13.190000\du,8.050000\du)--cycle;
% setfont left to latex
\definecolor{dialinecolor}{rgb}{0.000000, 0.000000, 0.000000}
\pgfsetstrokecolor{dialinecolor}
\node at (10.845000\du,9.195000\du){Button};
\definecolor{dialinecolor}{rgb}{1.000000, 1.000000, 1.000000}
\pgfsetfillcolor{dialinecolor}
\fill (8.400000\du,14.850000\du)--(8.400000\du,18.150000\du)--(13.550000\du,18.150000\du)--(13.550000\du,14.850000\du)--cycle;
\pgfsetlinewidth{0.100000\du}
\pgfsetdash{}{0pt}
\pgfsetdash{}{0pt}
\pgfsetmiterjoin
\definecolor{dialinecolor}{rgb}{0.000000, 0.000000, 0.000000}
\pgfsetstrokecolor{dialinecolor}
\draw (8.400000\du,14.850000\du)--(8.400000\du,18.150000\du)--(13.550000\du,18.150000\du)--(13.550000\du,14.850000\du)--cycle;
% setfont left to latex
\definecolor{dialinecolor}{rgb}{0.000000, 0.000000, 0.000000}
\pgfsetstrokecolor{dialinecolor}
\node at (10.975000\du,16.295000\du){Microphone};
% setfont left to latex
\definecolor{dialinecolor}{rgb}{0.000000, 0.000000, 0.000000}
\pgfsetstrokecolor{dialinecolor}
\node at (10.975000\du,17.095000\du){FIFO};
\definecolor{dialinecolor}{rgb}{1.000000, 1.000000, 1.000000}
\pgfsetfillcolor{dialinecolor}
\fill (23.050000\du,9.550000\du)--(23.050000\du,12.800000\du)--(27.400000\du,12.800000\du)--(27.400000\du,9.550000\du)--cycle;
\pgfsetlinewidth{0.100000\du}
\pgfsetdash{}{0pt}
\pgfsetdash{}{0pt}
\pgfsetmiterjoin
\definecolor{dialinecolor}{rgb}{0.000000, 0.000000, 0.000000}
\pgfsetstrokecolor{dialinecolor}
\draw (23.050000\du,9.550000\du)--(23.050000\du,12.800000\du)--(27.400000\du,12.800000\du)--(27.400000\du,9.550000\du)--cycle;
% setfont left to latex
\definecolor{dialinecolor}{rgb}{0.000000, 0.000000, 0.000000}
\pgfsetstrokecolor{dialinecolor}
\node at (25.225000\du,10.970000\du){State};
% setfont left to latex
\definecolor{dialinecolor}{rgb}{0.000000, 0.000000, 0.000000}
\pgfsetstrokecolor{dialinecolor}
\node at (25.225000\du,11.770000\du){Machine};
\definecolor{dialinecolor}{rgb}{1.000000, 1.000000, 1.000000}
\pgfsetfillcolor{dialinecolor}
\fill (28.600000\du,10.175000\du)--(28.600000\du,12.075000\du)--(31.812500\du,12.075000\du)--(31.812500\du,10.175000\du)--cycle;
\pgfsetlinewidth{0.100000\du}
\pgfsetdash{}{0pt}
\pgfsetdash{}{0pt}
\pgfsetmiterjoin
\definecolor{dialinecolor}{rgb}{0.000000, 0.000000, 0.000000}
\pgfsetstrokecolor{dialinecolor}
\draw (28.600000\du,10.175000\du)--(28.600000\du,12.075000\du)--(31.812500\du,12.075000\du)--(31.812500\du,10.175000\du)--cycle;
% setfont left to latex
\definecolor{dialinecolor}{rgb}{0.000000, 0.000000, 0.000000}
\pgfsetstrokecolor{dialinecolor}
\node at (30.206250\du,11.320000\du){Lamps};
\definecolor{dialinecolor}{rgb}{1.000000, 1.000000, 1.000000}
\pgfsetfillcolor{dialinecolor}
\fill (16.100000\du,5.500000\du)--(16.100000\du,16.850000\du)--(21.800000\du,16.850000\du)--(21.800000\du,5.500000\du)--cycle;
\pgfsetlinewidth{0.100000\du}
\pgfsetdash{}{0pt}
\pgfsetdash{}{0pt}
\pgfsetmiterjoin
\definecolor{dialinecolor}{rgb}{0.000000, 0.000000, 0.000000}
\pgfsetstrokecolor{dialinecolor}
\draw (16.100000\du,5.500000\du)--(16.100000\du,16.850000\du)--(21.800000\du,16.850000\du)--(21.800000\du,5.500000\du)--cycle;
% setfont left to latex
\definecolor{dialinecolor}{rgb}{0.000000, 0.000000, 0.000000}
\pgfsetstrokecolor{dialinecolor}
\node at (18.950000\du,11.370000\du){};
\definecolor{dialinecolor}{rgb}{1.000000, 1.000000, 1.000000}
\pgfsetfillcolor{dialinecolor}
\fill (16.700000\du,7.950000\du)--(16.700000\du,9.850000\du)--(21.200000\du,9.850000\du)--(21.200000\du,7.950000\du)--cycle;
\pgfsetlinewidth{0.100000\du}
\pgfsetdash{}{0pt}
\pgfsetdash{}{0pt}
\pgfsetmiterjoin
\definecolor{dialinecolor}{rgb}{0.000000, 0.000000, 0.000000}
\pgfsetstrokecolor{dialinecolor}
\draw (16.700000\du,7.950000\du)--(16.700000\du,9.850000\du)--(21.200000\du,9.850000\du)--(21.200000\du,7.950000\du)--cycle;
% setfont left to latex
\definecolor{dialinecolor}{rgb}{0.000000, 0.000000, 0.000000}
\pgfsetstrokecolor{dialinecolor}
\node at (18.950000\du,9.095000\du){Time};
\definecolor{dialinecolor}{rgb}{1.000000, 1.000000, 1.000000}
\pgfsetfillcolor{dialinecolor}
\fill (16.605000\du,10.300000\du)--(16.605000\du,12.200000\du)--(21.295000\du,12.200000\du)--(21.295000\du,10.300000\du)--cycle;
\pgfsetlinewidth{0.100000\du}
\pgfsetdash{}{0pt}
\pgfsetdash{}{0pt}
\pgfsetmiterjoin
\definecolor{dialinecolor}{rgb}{0.000000, 0.000000, 0.000000}
\pgfsetstrokecolor{dialinecolor}
\draw (16.605000\du,10.300000\du)--(16.605000\du,12.200000\du)--(21.295000\du,12.200000\du)--(21.295000\du,10.300000\du)--cycle;
% setfont left to latex
\definecolor{dialinecolor}{rgb}{0.000000, 0.000000, 0.000000}
\pgfsetstrokecolor{dialinecolor}
\node at (18.950000\du,11.445000\du){Button};
\definecolor{dialinecolor}{rgb}{1.000000, 1.000000, 1.000000}
\pgfsetfillcolor{dialinecolor}
\fill (16.402500\du,12.400000\du)--(16.402500\du,14.300000\du)--(21.497500\du,14.300000\du)--(21.497500\du,12.400000\du)--cycle;
\pgfsetlinewidth{0.100000\du}
\pgfsetdash{}{0pt}
\pgfsetdash{}{0pt}
\pgfsetmiterjoin
\definecolor{dialinecolor}{rgb}{0.000000, 0.000000, 0.000000}
\pgfsetstrokecolor{dialinecolor}
\draw (16.402500\du,12.400000\du)--(16.402500\du,14.300000\du)--(21.497500\du,14.300000\du)--(21.497500\du,12.400000\du)--cycle;
% setfont left to latex
\definecolor{dialinecolor}{rgb}{0.000000, 0.000000, 0.000000}
\pgfsetstrokecolor{dialinecolor}
\node at (18.950000\du,13.545000\du){Light Sensor};
\definecolor{dialinecolor}{rgb}{1.000000, 1.000000, 1.000000}
\pgfsetfillcolor{dialinecolor}
\fill (16.536250\du,14.650000\du)--(16.536250\du,16.550000\du)--(21.363750\du,16.550000\du)--(21.363750\du,14.650000\du)--cycle;
\pgfsetlinewidth{0.100000\du}
\pgfsetdash{}{0pt}
\pgfsetdash{}{0pt}
\pgfsetmiterjoin
\definecolor{dialinecolor}{rgb}{0.000000, 0.000000, 0.000000}
\pgfsetstrokecolor{dialinecolor}
\draw (16.536250\du,14.650000\du)--(16.536250\du,16.550000\du)--(21.363750\du,16.550000\du)--(21.363750\du,14.650000\du)--cycle;
% setfont left to latex
\definecolor{dialinecolor}{rgb}{0.000000, 0.000000, 0.000000}
\pgfsetstrokecolor{dialinecolor}
\node at (18.950000\du,15.795000\du){Microphone};
\definecolor{dialinecolor}{rgb}{1.000000, 1.000000, 1.000000}
\pgfsetfillcolor{dialinecolor}
\fill (16.275000\du,5.600000\du)--(16.275000\du,7.550000\du)--(21.625000\du,7.550000\du)--(21.625000\du,5.600000\du)--cycle;
\pgfsetlinewidth{0.100000\du}
\pgfsetdash{}{0pt}
\pgfsetdash{}{0pt}
\pgfsetmiterjoin
\definecolor{dialinecolor}{rgb}{1.000000, 1.000000, 1.000000}
\pgfsetstrokecolor{dialinecolor}
\draw (16.275000\du,5.600000\du)--(16.275000\du,7.550000\du)--(21.625000\du,7.550000\du)--(21.625000\du,5.600000\du)--cycle;
% setfont left to latex
\definecolor{dialinecolor}{rgb}{0.000000, 0.000000, 0.000000}
\pgfsetstrokecolor{dialinecolor}
\node at (18.950000\du,6.770000\du){Sensor struct};
\pgfsetlinewidth{0.100000\du}
\pgfsetdash{}{0pt}
\pgfsetdash{}{0pt}
\pgfsetbuttcap
{
\definecolor{dialinecolor}{rgb}{0.000000, 0.000000, 0.000000}
\pgfsetfillcolor{dialinecolor}
% was here!!!
\pgfsetarrowsend{latex}
\definecolor{dialinecolor}{rgb}{0.000000, 0.000000, 0.000000}
\pgfsetstrokecolor{dialinecolor}
\draw (13.200000\du,6.650000\du)--(16.700000\du,8.900000\du);
}
\pgfsetlinewidth{0.100000\du}
\pgfsetdash{}{0pt}
\pgfsetdash{}{0pt}
\pgfsetbuttcap
{
\definecolor{dialinecolor}{rgb}{0.000000, 0.000000, 0.000000}
\pgfsetfillcolor{dialinecolor}
% was here!!!
\pgfsetarrowsend{latex}
\definecolor{dialinecolor}{rgb}{0.000000, 0.000000, 0.000000}
\pgfsetstrokecolor{dialinecolor}
\draw (13.190000\du,9.000000\du)--(16.605000\du,11.250000\du);
}
\pgfsetlinewidth{0.100000\du}
\pgfsetdash{}{0pt}
\pgfsetdash{}{0pt}
\pgfsetbuttcap
{
\definecolor{dialinecolor}{rgb}{0.000000, 0.000000, 0.000000}
\pgfsetfillcolor{dialinecolor}
% was here!!!
\pgfsetarrowsend{latex}
\definecolor{dialinecolor}{rgb}{0.000000, 0.000000, 0.000000}
\pgfsetstrokecolor{dialinecolor}
\draw (13.492500\du,11.250000\du)--(16.402500\du,13.350000\du);
}
\pgfsetlinewidth{0.100000\du}
\pgfsetdash{}{0pt}
\pgfsetdash{}{0pt}
\pgfsetbuttcap
{
\definecolor{dialinecolor}{rgb}{0.000000, 0.000000, 0.000000}
\pgfsetfillcolor{dialinecolor}
% was here!!!
\pgfsetarrowsend{latex}
\definecolor{dialinecolor}{rgb}{0.000000, 0.000000, 0.000000}
\pgfsetstrokecolor{dialinecolor}
\draw (21.800000\du,11.175000\du)--(23.050000\du,11.175000\du);
}
\pgfsetlinewidth{0.100000\du}
\pgfsetdash{}{0pt}
\pgfsetdash{}{0pt}
\pgfsetbuttcap
{
\definecolor{dialinecolor}{rgb}{0.000000, 0.000000, 0.000000}
\pgfsetfillcolor{dialinecolor}
% was here!!!
\pgfsetarrowsend{latex}
\definecolor{dialinecolor}{rgb}{0.000000, 0.000000, 0.000000}
\pgfsetstrokecolor{dialinecolor}
\draw (27.400000\du,11.175000\du)--(28.600000\du,11.125000\du);
}
\pgfsetlinewidth{0.100000\du}
\pgfsetdash{}{0pt}
\pgfsetdash{}{0pt}
\pgfsetmiterjoin
\pgfsetbuttcap
{
\definecolor{dialinecolor}{rgb}{0.000000, 0.000000, 0.000000}
\pgfsetfillcolor{dialinecolor}
% was here!!!
\pgfsetarrowsend{latex}
{\pgfsetcornersarced{\pgfpoint{0.000000\du}{0.000000\du}}\definecolor{dialinecolor}{rgb}{0.000000, 0.000000, 0.000000}
\pgfsetstrokecolor{dialinecolor}
\draw (8.580949\du,13.500000\du)--(7.350000\du,13.500000\du)--(7.350000\du,16.500000\du)--(8.400000\du,16.500000\du);
}}
\pgfsetlinewidth{0.100000\du}
\pgfsetdash{}{0pt}
\pgfsetdash{}{0pt}
\pgfsetmiterjoin
\pgfsetbuttcap
{
\definecolor{dialinecolor}{rgb}{0.000000, 0.000000, 0.000000}
\pgfsetfillcolor{dialinecolor}
% was here!!!
\pgfsetarrowsend{latex}
{\pgfsetcornersarced{\pgfpoint{0.000000\du}{0.000000\du}}\definecolor{dialinecolor}{rgb}{0.000000, 0.000000, 0.000000}
\pgfsetstrokecolor{dialinecolor}
\draw (13.550000\du,16.500000\du)--(15.043125\du,16.500000\du)--(15.043125\du,15.600000\du)--(16.536250\du,15.600000\du);
}}
\end{tikzpicture}
}
%\caption{Diagrama de Flujo de datos}
%\label{fig:data_flow}
%\end{figure}


%----------------------
\textbf{ Bibliografía}
% ----------------------

Technical Reference manual, MSP432P4XX Simple Link Microcontrollers. Texas Instruments. Marzo 2017.

MSP432 Peripheral Driver Library. Users Guide. Texas Instruments. 2015.

BOOSTXL-EDUMKII Educational BoosterPack, Module Mark II. Texas Instruments. Marzo 2017.

Zazu kids voice activated night licht. Amazon.com Tomado 18 setiembre 2017.

%\bibliographystyle{apacite}
%\bibliography{../general/biblografia.bib}
%---------------------------------------------------------------------------------------------------------------
\end{document}